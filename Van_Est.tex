\pdfoutput=1
%The other issue is that some packages, such as microtype, produce different output under pdflatex. By default the arXiv goes from dvi to ps to pdf, so if you need pdflatex you have to set the \pdfoutput flag in the TeX file.
\newif\ifpersonal
\personaltrue % comment to remove personal notes

\documentclass[10pt,a4paper]{amsart}
\usepackage{fullpage}
%\linespread{1.1}
%\usepackage{amsmath, amscd, amssymb, amsthm, latexsym, url, color, todonotes} %pdflscape}
%\usepackage{graphicx}
\usepackage{mathrsfs}
\usepackage[left=1.0in, right=1.0in, top=1in, bottom=1.3in, includefoot, headheight=13.6pt]{geometry}
\usepackage[colorlinks=true,hyperindex,citecolor=blue,linkcolor=black]{hyperref}
\input{xy}
%\cXyoption{all}
%\usepackage{natbib}
\usepackage{cite}
\usepackage{tikz-cd}
%\usepackage{stix}
\setlength{\marginparwidth}{1in}
\usepackage[capitalize]{cleveref}
%\usepackage[utf8]{inputenc}
\usepackage{eucal,times,amsmath,amsthm,amssymb,mathrsfs,stmaryrd,color,enumerate,accents}
\usepackage{tensor}

\usepackage{textcomp}
%\RequirePackage[l2tabu,orthodox]{nag} %detect whether obsolete packages are used
%\documentclass[10pt,a4paper,reqno]{amsart} %reqno places equation numbers on the right
%\linespread{1.1}
\usepackage{mathtools,bm,eucal} % math related
%\usepackage{microtype,fixltx2e,lmodern} % latex technical issues
%\usepackage[utf8]{inputenc} % input encoding
\usepackage[T1]{fontenc} % font encoding
\usepackage{enumerate,comment,braket,xspace,csquotes} % utilities
%\usepackage[all,cmtip]{xy} % because the tikzcd options [shift left], [shift right] do not work on arXiv, we switched some diagrams to xymatrix
%\usepackage[centering,vscale=0.7,hscale=0.8]{geometry}
%\usepackage[hidelinks]{hyperref}

\numberwithin{equation}{subsection}
\newcommand{\heart}{\ensuremath\heartsuit}
%BEGIN MY COMMANDS
\newtheorem{theorem}{Theorem}[subsection]
\newtheorem{lemma}[theorem]{Lemma}
\newtheorem{corollary}[theorem]{Corollary}
\newtheorem{proposition}[theorem]{Proposition}
\newtheorem{conjecture}[theorem]{Conjecture}
\newtheorem{question}[theorem]{Question}
\newtheorem{assumption}[theorem]{Assumption}
\newtheorem{claim}[theorem]{Claim}
\newtheorem{thm-intro}{Theorem}

\newtheorem{ap-theorem}{Theorem}
\newtheorem{ap-corollary}{Corollary}
\newtheorem{ap-lemma}{Lemma}
\newtheorem{ap-proposition}{Proposition}
\newtheorem{ap-definition}{Definition}


\theoremstyle{definition}
\newtheorem{remark}[theorem]{Remark}
\newtheorem{example}[theorem]{Example}
\newtheorem{notation}[theorem]{Notation}
\newtheorem{definition}[theorem]{Definition}
\newtheorem{construction}[theorem]{Construction}
\newtheorem{warning}[theorem]{Warning}
\newtheorem{convention}[theorem]{Convention}
%\newtheorem{exercise}[theorem]{Exercise}

\renewcommand{\labelenumi}{(\roman{enumi})}
\renewcommand{\labelitemi}{-}

\newcommand{\cXysquare}[8]{
\[\cXymatrix{
#1 \ar@{#5}[r] \ar@{#6}[d] & #2 \ar@{#7}[d]\\
#3 \ar@{#8}[r] & #4
}\]
}

\DeclareMathOperator*{\indlimf}{``\colim''}
\DeclareMathOperator*{\projlimf}{``\lim''}
\DeclareMathOperator*{\holim}{\operatorname*{holim}}
\def\lim{\mathrm{lim}}
\DeclareMathOperator*{\hocolim}{\operatorname*{hocolim}}
\DeclareRobustCommand{\SkipTocEntry}[5]{}


%\newcommand{\abs}[1]{\vert#1\vert}

\ifpersonal
\newcommand{\personal}[1]{\textcolor[rgb]{0,0,1}{(Personal: #1)}}
\newcommand{\todo}[1]{\textcolor{red}{(Todo: #1)}}
\else
\newcommand*{\personal}[1]{\ignorespaces}
\newcommand*{\todo}[1]{\ignorespaces}
\fi

\newcommand{\abs}[1]{\vert#1\vert}


% Z_p-adic geometry
\newcommand{\bZp}{\bZ_p}
\newcommand{\adSt}{\mathrm{dSt}^\ad}
\newcommand{\cycbZp}{\bZp^\cyc}
\newcommand{\Ainf}{\mathrm{A}_{\mathrm{inf}}}
\newcommand{\AdR}{\mathrm{A}_{\dR}}


%groups
\newcommand{\GL}{\mathrm{GL}}
\newcommand{\GLn}{\mathrm{GL}_n}
\newcommand{\Gk}{G_K}
\newcommand{\Gal}{\mathrm{Gal}}
\newcommand{\anGLn}{\mathbf{GL}_n^\an}

%stacks
\newcommand{\fib}{\mathrm{fib}}
\newcommand{\dLocSys}{\mathrm R \mathrm{LocSys}_{\ell, n}}
\newcommand{\abdLocSys}{\mathrm R \mathrm{LocSys}_{\ell, n, \Gamma}}
\newcommand{\LocSys}{\mathrm{LocSys}_{\ell, n}}
\newcommand{\abLocSys}{\mathrm{LocSys}_{\ell, n, \Gamma}}
\newcommand{\LocSysfr}{\LocSys^{\mathrm{framed}}}
\newcommand{\abLocSysfr}{\abLocSys^{\mathrm{framed}}}
\newcommand{\cLocSys}{\mathrm{LocSys}_{\bC, n}}
\newcommand{\Conn}{\mathrm{Conn}_{\bC, n }}
\newcommand{\Hilb}{\mathrm{Hilb}}
\newcommand{\Vect}{\mathrm{Vect}}
\newcommand{\bG}{\rmB G}

%symbols
\newcommand{\FormalModels}{\mathrm{FM}}
\newcommand{\Idem}{\mathrm{Idem}}
\newcommand{\idem}{\mathrm{idem}}
\newcommand{\nil}{\mathrm{nil}}
\newcommand{\cofib}{\mathrm{cofib}}
\newcommand{\id}{\mathrm{Id}}
\newcommand{\fr}{\widehat{\rmF}_r}
\newcommand{\Higgs}{\mathrm{Higgs}}
\newcommand{\ev}{\mathrm{ev}}
\newcommand{\sm}{\mathrm{sm}}
\newcommand{\tr}{\mathrm{tr}}
%\newcommand{\dim}{\mathrm{dim}}
\newcommand{\Frob}{\mathrm{Frob}}
\newcommand{\barX}{\overline{X}}
\newcommand{\pr}{\mathrm{pr}_1}
\newcommand{\Map}{\mathrm{Map}}
\newcommand{\fpqc}{\mathrm{fpqc}}
\newcommand{\Hom}{\mathrm{Hom}}
\newcommand{\ad}{\mathrm{ad}}
\newcommand{\Ex}{\mathrm{Ex}}
\newcommand{\fCoh}{\mathfrak{Coh}^+}
\newcommand{\cyc}{\mathrm{cyc}}
\newcommand{\inf}{\mathrm{inf}}

\newcommand{\cotimes}{\widehat{\otimes}}
\newcommand{\sX}{\mathscr{X}}
\newcommand{\sY}{\mathscr{Y}}
\newcommand{\sZ}{\mathscr{Z}}

\newcommand{\sfX}{\mathsf{X}}
\newcommand{\sfY}{\mathsf{Y}}
\newcommand{\sfZ}{\mathsf{Z}}
\newcommand{\sfW}{\mathsf{W}}

\newcommand{\Ok}{k^\circ}


\DeclareMathOperator{\Spec}{Spec}
\DeclareMathOperator{\Spf}{Spf}
\DeclareMathOperator{\Sp}{Sp}
%\newcommand{\Spf}{\mathrm{Spf}}
%\newcommand{\Spec}{\mathrm{Spec}}
\newcommand{\st}{\mathrm{st}}
\newcommand{\Der}{\mathrm{Der}}
\newcommand{\trun}{\mathrm{t}}
\newcommand{\spe}{\mathrm{sp}}

\newcommand{\bfA}{\mathfrak{A}_{\Ok}}
\newcommand{\anA}{\mathbf{A}_k}
\newcommand{\anB}{\mathbf{B}_k}
\newcommand{\banA}{\mathbf{A}_{\Ok}}

\newcommand{\cEnd}{\cE \mathrm{nd}}
\newcommand{\End}{\mathrm{End}}
\newcommand{\Mat}{\mathrm{Mat}}
\newcommand{\unr}{\mathrm{unr}}
\newcommand{\Frac}{\mathrm{Frac}}


\newcommand{\tamepi}{\pi_1^{\mathrm{t}}}
\newcommand{\wildpi}{\pi_1^w}
\newcommand{\tame}{\mathrm{tame}}
\newcommand{\Mon}{\mathrm{Mon}}
\newcommand{\grp}{\mathrm{grp}}
\newcommand{\dSt}{\mathrm{dSt}}

\newcommand{\fc}{\mathrm{fc}}
\newcommand{\Sh}{\mathrm{Sh}}
\newcommand{\Ad}{\mathrm{Ad}}

\newcommand{\Def}{\mathbf{\mathrm{Def}}}
\newcommand{\art}{\mathrm{art}}
\newcommand{\cont}{\mathrm{cont}}
\newcommand{\Q}{\mathbb{Q}}
\newcommand{\Ql}{\bQ_\ell}
\newcommand{\adL}{\mathbb{L}^{\ad}}

\newcommand{\Aut}{\mathrm{Aut}}
%homotopy types


%pregeometries and local structures and algebraic categories

\newcommand{\Tdisc}{\mathcal{T}_{\mathrm{disc}}}
\newcommand{\Tet}{\mathcal{T}_{\text{\'et}}}
\newcommand{\Tetph}{\mathcal{T}_{\emph{\text{\'et}}}}
\newcommand{\Tzar}{\mathcal{T}_{\mathrm{Zar}}}
\newcommand{\Tan}{\mathcal{T}_{\an}(k)}
\newcommand{\Tad}{\mathcal{T}_{\ad}(\Ok)}
\newcommand{\cTad}{\mathcal{T}_{\ad}(\Ok)}
\newcommand{\sm}{\mathcal{C}\mathrm{Alg}^\sm}
\newcommand{\CAlg}{\mathcal{C}\mathrm{Alg}}
\newcommand{\adCAlg}{\cC \mathrm{Alg}^{\ad}_{\bZp}}
\newcommand{\CAlgad}{\cC \mathrm{Alg}^{\ad}}
\newcommand{\admCAlg}{\cC \mathrm{Alg}^{\mathrm{adm}}_{\Ok}}
\newcommand{\fCAlg}{\mathrm{f} \cC \mathrm{Alg}_{\Ok}}
\newcommand{\ftaftCAlg}{\fCAlg^{\taft}}
\newcommand{\taft}{\mathrm{taft}}
\newcommand{\AnRing}{\mathrm{AnRing}_k}
\newcommand{\wCAlg}{\cC \mathrm{Alg}^{\wedge}_{\Ok}}



\newcommand{\Str}{\mathrm{Str}}
\newcommand{\locStr}{\mathrm{Str}^{\mathrm{loc}}}
\newcommand{\adm}{\mathrm{adm}}


\newcommand{\PSh}{\mathrm{PSh}}
\newcommand{\PrL}{\mathcal{P} \mathrm{r}^{\mathrm L}}
\newcommand{\Cat}{\mathcal{C}\mathrm{at}_\infty}
\newcommand{\Coh}{\mathrm{Coh}^+}
\newcommand{\Coheart}{\mathrm{Coh}^{+, \heartsuit}}
\newcommand{\Cohb}{\mathrm{Coh}^{\mathrm b}}
\newcommand{\bCoh}{\mathrm{Coh}^\mathrm{b}}
\newcommand{\cHom}{\mathcal{H} \mathrm{om}}
\newcommand{\loc}{\mathrm{loc}}



%infty categories, spaces, modules
\newcommand{\op}{\mathrm{op}}
\newcommand{\der}{\mathrm{der}}
\newcommand{\fSpAb}{\mathrm{Sp} \left( \mathrm{Ab}( \fstrfs) \right) }
\newcommand{\Psh}{\mathrm{PShv}}
\newcommand{\Shv}{\mathrm{Shv}}
\newcommand{\ind}{\mathrm{Ind}}
\newcommand{\Ind}{\mathrm{Ind}}
\newcommand{\pro}{\mathrm{Pro}}
\newcommand{\Perf}{\mathrm{Perf}}
\newcommand{\DM}{Deligne-Mumford stack}
\newcommand{\inftopos}{$\infty$-topos\xspace}
\newcommand{\infcat}{$\infty$-category\xspace}
\newcommand{\infcats}{$\infty$-categories\xspace}
\newcommand{\sMap}{\underline{\Map}}




\newcommand{\PerfSys}{\mathrm{PerfSys}_{\ell}}
\newcommand{\rigCat}{\mathcal{C}\mathrm{at}^{\mathrm{st}, \omega, \otimes}_{\infty}}



%cohomology
\newcommand{\coho}{C^*_{\mathrm{cont}}( \K, \mathrm{Ad}(\rho))}\newcommand{\cohon}{C^*_{\mathrm{cont}}( \K, \mathrm{Ad}(\rho_n))}
\newcommand{\proet}{\text{pro\'et}}
\newcommand{\et}{\text{\'et}}
\newcommand{\emphet}{\emph{\text{\'et}}}
\newcommand{\Sym}{\mathrm{Sym}}
\newcommand{\dR}{\mathrm{dR}}

%functors
\newcommand{\tcomp}{(-)^\wedge_t}
\newcommand{\functor}{(-)}
\newcommand{\rigg}{(-)^{\mathrm{rig}}}
\newcommand{\rig}{\mathrm{rig}}
\newcommand{\alg}{\mathrm{alg}}
\newcommand{\Fun}{\mathrm{Fun}}
\newcommand{\Hub}{(-)^+}
\newcommand{\sh}{\mathrm{sh}}
\newcommand{\disc}{\functor^\mathrm{disc}}


%for categories of geometric objects
\newcommand{\Top}{\tensor[^{\mathrm{R}}]{\cT \op}{}}
\newcommand{\TopT}{\tensor[^{\mathrm{R}}]{\cT \op}{}( \Tau)}
\newcommand{\adTop}{\tensor[^{\mathrm{R}}]{\cT \op}{}( \Tad)}
\newcommand{\anTop}{\tensor[^{\mathrm{R}}]{\cT \op}{}( \Tan)}
\newcommand{\etTop}{\tensor[^{\mathrm{R}}]{\cT \op}{} (\Tet(\Ok))}
\newcommand{\etphTop}{\tensor[^{\mathrm{R}}]{\cT \op}{} (\Tetph(\Ok))}
\newcommand{\etTopk}{\tensor[^{\mathrm{R}}]{\cT \op}{}(\Tet(k))}
\newcommand{\discTop}{\tensor[^{\mathrm{R}}]{\cT \op}{}(\Tdisc(\Ok))}
\newcommand{\discTopk}{\tensor[^{\mathrm{R}}]{\cT \op}{}(\Tdisc(k))}
\newcommand{\discTopn}{\tensor[^{\mathrm{R}}]{\cT \op}{} (\Tdisc(\Ok_n))}

\newcommand{\Mod}{\mathrm{Mod}}
\newcommand{\St}{\mathrm{St}}

\newcommand{\Afd}{\mathrm{Afd}}
\newcommand{\Afdl}{\mathrm{Afd}_{\Q_\ell}}
\newcommand{\An}{\mathrm{An}}
\newcommand{\Anl}{\mathrm{An}_{\Q_\ell}}
\newcommand{\dAnl}{\mathrm{dAn}_{\Q_\ell}}
\newcommand{\dAfd}{\mathrm{dAfd}}
\newcommand{\dAfdl}{\mathrm{dAfd}_{\Q_\ell}}
\newcommand{\dAn}{\mathrm{dAn}}
\newcommand{\Aff}{\mathrm{Aff}}
\newcommand{\dAff}{\mathrm{dAff}}
\newcommand{\dSch}{\mathrm{dSch}_{\bZp}}
\newcommand{\Sch}{\mathrm{Sch}_k}
\newcommand{\dfSch}{\mathrm{dfSch}}
\newcommand{\fSch}{\mathrm{fSch}_{\Ok}}
\newcommand{\dfAff}{\mathrm{dfAff}_{\Ok}}
\newcommand{\dfDM}{\mathrm{dfDM}}
\newcommand{\fDM}{\mathrm{fDM}_{\Ok}}
\newcommand{\an}{\mathrm{an}}
%\newcommand{\Sp}{\mathrm{Sp}}
\newcommand{\Ab}{\mathrm{Ab}}
%\newcommand{\Set}{\mathrm{Set}}

%rings
\newcommand{\Zl}{\mathbb{Z}_{\ell}}


\DeclareMathOperator*{\colim}{colim}

% \mathrm
\newcommand{\rmA}{\mathrm{A}}
\newcommand{\rmB}{\mathrm{B}}
\newcommand{\rmC}{\mathrm C}
\newcommand{\rmD}{\mathrm D}
\newcommand{\rmE}{\mathrm E}
\newcommand{\rmF}{\mathrm F}
\newcommand{\rmG}{\mathrm G}
\newcommand{\rmH}{\mathrm H}
\newcommand{\rmI}{\mahtrm I}
\newcommand{\rmJ}{\mathrm J}
\newcommand{\rmL}{\mathrm{L}}
\newcommand{\rmM}{\mathrm M}
\newcommand{\rmP}{\mathrm P}
\newcommand{\rmR}{\mathrm R}
\newcommand{\rmT}{\mathrm T}

% \mathbb
\newcommand{\bA}{\mathbb A}
\newcommand{\bB}{\mathbb B}
\newcommand{\bC}{\mathbb C}
\newcommand{\bD}{\mathbb D}
\newcommand{\bE}{\mathbb E}
\newcommand{\bF}{\mathbb F}
\newcommand{\bL}{\mathbb L}
\newcommand{\bP}{\mathbb P}
\newcommand{\bN}{\mathbb N}
\newcommand{\bQ}{\mathbb Q}
\newcommand{\bR}{\mathbb{R}}
\newcommand{\bT}{\mathbb T}
\newcommand{\bZ}{\mathbb Z}

%\mathfrak
\newcommand{\fA}{\mathfrak A}
\newcommand{\fB}{\mathfrak B}
\newcommand{\fC}{\mathfrak C}
\newcommand{\fF}{\mathfrak F}
\newcommand{\fG}{\mathfrak G}
\newcommand{\ff}{\mathfrak f}
\newcommand{\fU}{\mathfrak U}
\newcommand{\fX}{\mathfrak X}
\newcommand{\fY}{\mathfrak Y}
\newcommand{\fZ}{\mathfrak Z}
\newcommand{\fg}{\mathfrak g}
\newcommand{\fm}{\mathfrak{m}}
\newcommand{\fl}{\mathfrak l}

%\mathcal
\newcommand{\cA}{\mathcal A}
\newcommand{\cB}{\mathcal B}
\newcommand{\cC}{\mathcal C}
\newcommand{\cD}{\mathcal D}
\newcommand{\cE}{\mathcal E}
\newcommand{\cF}{\mathcal{F}}
\newcommand{\cG}{\mathcal{G}}
\newcommand{\cI}{\mathcal{I}}
\newcommand{\cH}{\mathcal{H}}
\newcommand{\cK}{\mathcal{K}}
\newcommand{\cL}{\mathcal{L}}
\newcommand{\cM}{\mathcal M}
\newcommand{\cN}{\mathcal N}
\newcommand{\cO}{\mathcal{O}}
\newcommand{\cP}{\mathcal{P}}
\newcommand{\cQ}{\mathcal{Q}}
\newcommand{\cV}{\mathcal V}
\newcommand{\cT}{\mathcal{T}}
\newcommand{\cX}{\mathcal X}
\newcommand{\cY}{\mathcal Y}
\newcommand{\cZ}{\mathcal Z}
\newcommand{\cS}{\mathcal S}

%\widehat

\newcommand{\hB}{\widehat{B}}
\newcommand{\hAA}{\widehat{A'}}
\newcommand{\hAp}{\widehat{A}_X[p^{-1}]}
\newcommand{\hbZ}{\widehat{\bZ}}
%\newcommand{\abs}[1]{\vert#1\vert}
\newcommand{\overK}{\overline{K}}

%\widetilde

\newcommand{\tP}{\widetilde{P}}
\newcommand{\tQ}{\widetilde{Q}}
\newcommand{\tR}{\widetilde{R}}
\newcommand{\tS}{\widetilde{S}}
\newcommand{\tT}{\widetilde{T}}
\newcommand{\tV}{\widetilde{V}}
\newcommand{\tW}{\widetilde{W}}
\newcommand{\tX}{\widetilde{X}}
\newcommand{\tY}{\widetilde{Y}}
\newcommand{\tZ}{\widetilde{Z}}

%\Cech
\newcommand{\vC}{\textrm{\v{C}}}




\author{Jorge ANT\'ONIO}
\address{Jorge Ant\'onio,  Institut de Math\'ematiques de Toulouse, 118 Rue de Narbonne  31400 Toulouse}
\email{\texttt{jorge\_tiago.ferrera\_antonio@math.univ-toulouse.fr}}

\author{Joost NUITEN}
\address{Joost Nuiten, IMAG Universit\'e de Montpellier, Place Eug\`{e}ne Bataillon
34090 Montpellier}

\date{}

\title{A Van Est theorem in mixed characteristic} 



\begin{document}


\maketitle

\personal{PERSONAL COMMENTS ARE SHOWN!!!}

\tableofcontents

\section{Geometric context}
\todo{Update the HA bib entry and add Fontaine's reference for the fact that $\Ainf(-)$ provides a pro-universal thickening of perfectoid!}

\subsection{Derived $\bZp$-adic geometric stacks} In this \S we give an overview of the construction of the \infcat of derived $\bZp$-adic geometric stacks.

\begin{definition}
    Let $\CAlg^\ad$ denote the \infcat defined via the pullback square
        \[
        \begin{tikzcd}
            \CAlg^\ad \ar{r} \ar{d} & \CAlg \ar{d}{\pi_0(-)} \\
            \CAlg^{\heartsuit, \ad} \ar{r} & \CAlg^\heartsuit
        \end{tikzcd}
        \]
    computed in the \infcat $\Cat$. We refer to $\CAlg^\ad$ as the \infcat of \emph{simplicial adic algebras}. Let $A \in \CAlg^\ad$ be a simplicial $A$-adic algebra, we define the \infcat of simplicial $A$-adic algebras assertion
        \[
                \CAlg^\ad_{A} \coloneqq (\CAlg^\ad)_{A/}  .
        \]
    Of principal importance for us is the \infcat $\adCAlg$, where we regard $\bZp \in \CAlg^{\heartsuit, \ad}$ equipped with its $(p)$-adic topology.
\end{definition}

Let $f \colon A \to B$ be a morphism of simplicial $\bZp$-adic algebras. The \emph{relative $\bZp$-adic cotangent complex}, denoted $\adL_{B/A} \in \Mod_B$, is defined in \cite[\S 3.4]{antonio2018p}. It corepresents $\bZp$-adic derivations of simplicial $\bZp$-adic algebras.
In particular, we have a natural equivalence of mapping spaces 
    \[
            \Map_{\Mod_B}(\adL_{B/A}, M) \simeq \Map_{\CAlg^{\ad}_{A/ / B}}(B , B \oplus M) \in \cS,
    \]  
for every $M \in \Mod_B$.

\begin{definition}
    Let $f \colon A \to B $ be a morphism in the \infcat $\adCAlg$ between $p$-complete simplicial $\bZp$-adic algebras. We say that $f$ is \emph{topologically almost of finite presentation} if
        \[
            \pi_0(f) \colon \pi_0(A) \to \pi_0(B)  
        \]
    is a topologically of finite presentation morphism between ordinary $p$-adic complete $\bZp$-adic algebras and, for every integer $i>0$,
        \[
                \pi_i(B) \in \Mod_{\pi_0(A)}  
        \]
    is a finitely presented $\pi_0(A)$-module.
\end{definition}


\begin{definition}
    Let $f \colon A \to B$ be a morphism in the \infcat $\adCAlg$. The morphism $f$ is said to be \emph{\'etale} (resp., \emph{smooth}) if it is \emph{almost topologically of finite presentation} and the relative cotangent complex
        \[
                \adL_{B/A} \in \Mod_B.   
        \]
    vanishes (resp. it is equivalent to a free $B$-module of finite rank concentrated in degree $0$).
\end{definition}


\begin{definition}
    Let $\rmP_\sm$ denote the class of smooth morphisms in the \infcat $\adCAlg$. The triplet $(\adCAlg, \tau_\et, \rmP_\sm)$ forms a \emph{geometric context}, which we refer to as the \emph{derived $\bZp$-adic geometric context}. For the definition of
    the notion of geometric stack we refer the reader to \cite[Definition 2.3.1]{antonio2017moduli}.
\end{definition}

\begin{remark}
    Notice that the conditions (ii) and (iii) in \cite[Definition 2.3.1]{antonio2017moduli} are automatic. Condition (iv) follows from the local structure for smooth morphisms and
    condition (i) (and furhter faithfully flat descent) follows from \cite[Proposition 3.2.9]{antonio2018p} together with \cite[Theorem 5.15]{Lurie_Spectral_Schemes} combined with \cite[Proposition 8.1.2.1]{lurie2016spectral}.
\end{remark}

\begin{definition}
    The \infcat of \emph{derived $\bZp$-adic geometric stacks} is defined as the full subcategory of $\Shv(\adCAlg, \tau_\et)$ spanned by geometric stacks with respect to the geometric context $(\adCAlg, \tau_\et, \rmP_\sm)$,
        \[
             \adSt \coloneqq \dSt( \adCAlg, \tau_\et, \rmP).
        \]
\end{definition}

\begin{remark}
    The \infcat $\Shv(\adCAlg, \tau_\et)$ is \emph{closed under $\tau_\et$-descent}. This assertion is a consequence of \cite[Proposition 5.16, (3)]{Lurie_Spectral_Schemes}.
\end{remark}

\begin{example}
    In \cite[Definition 3.2.10]{antonio2018p} it was introduced the notion of a \emph{derived $\bZp$-adic Deligne-Mumford stack}. The collection of such objects forms naturally an \infcat, denoted $\dfDM_{\bZp}$.
    Let $\sfX \in \dfDM_{\bZp}$, then its associated functor of points is naturally an object in the \infcat $\Shv(\adCAlg, \tau_\et)$ and even an object in the \infcat $\Shv(\adCAlg, \tau_\fpqc)$, thanks to \cite[Theorem 5.15]{Lurie_Spectral_Schemes} combined with \cite[Proposition 8.1.2.1]{lurie2016spectral}.
    Therefore the association 
        \[
            \sfX \in \dfDM_{\bZp} \mapsto \Map_{\dfDM}(\Spf(-), \sfX) \in \Shv(\adCAlg, \tau_\et)  
        \]
    provides us with a fully faithful functor 
        \[
            F \colon \dfDM_{\bZp} \subseteq \Shv(\adCAlg, \tau_\et).
        \]
    In particular, if $\sfX \in \dfSch_{\bZp}$ is a derived $\bZp$-adic scheme, then we can regard it naturally
    as an \'etale sheaf on $\adCAlg$.

    We further assert that the fully faithful functor $F \colon \dfDM_{\bZp} \to \Shv(\adCAlg, \tau_\et)$ factors through the fully faithful natural inclusion $\adSt \subseteq \Shv(\adCAlg, \tau_\et)$.
    Indeed, since the \infcat $\Shv(\adCAlg, \tau_\et)$ is closed under $\tau_\et$-descent we are reduced to verify condition (1) in \cite[Definition 2.3.3]{antonio2017moduli} for derived $\bZp$-adic Deligne-Mumford stacks. This last assertion is immediate from the fact that derived $\bZp$-adic Deligne-Mumford stacks admit
    (affine) \'etale coverings, by construction. We have thus a natural fully faithful functor
        \[
                G \colon \dfDM_{\bZp} \to \adSt.    
        \] 
\end{example}

\begin{construction}
    Let $
            L_p^\wedge \colon \CAlg_{\bZp} \to \adCAlg
        $
    denote the \emph{$p$-completion functor}, introduced in \cite[\S 8]{lurie2016spectral}. Given $A \in \adCAlg$, we define $A_n \in \adCAlg$ as the pushout
        \[
        \begin{tikzcd}
            A[t] \ar{r}{t \mapsto p^n} \ar{d}{t \mapsto 0} & A \ar{d} \\
            A \ar{r} & A_n
        \end{tikzcd}
        \]    
    computed in \infcat $\CAlg_{\bZp}$. Notice that $A_n$ is naturally an object of the \infcat $\adCAlg$ via the canonical inclusion $\adCAlg \subseteq \CAlg_{\bZp}$.
    Thanks to \cite[Lemma 8.1.2.3]{lurie2016spectral}, one has a natural equivalence
        \[
            (A)^\wedge_p \simeq \underset{n\geq 1}{\lim} A_n,  
        \]
    in the \infcat $\adCAlg$, where $(A)^\wedge_p$ denotes the $p$-completion of $A$.
\end{construction}

\begin{remark}
    Given a functor $X \colon \CAlg_{\bZp} \to \cS$, we define its \emph{$p$-completion} as the functor
        \[
                X^\wedge_p \colon \adCAlg \mapsto \cS  
        \]
    given by the formula
        \begin{equation} \label{p-comp_stacks}
                A \in \adCAlg \mapsto \lim_{n \geq 1} X(A_n) \in \cS.  
        \end{equation}
    From the above formula \eqref{p-comp_stacks}, it is clear that if $X$ satisfies \'etale hyper-descent then so it does $X^\wedge_p$ satisfies descent with respect to $\tau_\et$-hypercoverings in $(\adCAlg, \tau_\et)$.
    The same statement holds if we replace \'etale hyper-descent by fpqc hyper-descent.
\end{remark}

\begin{lemma} \label{const_of_adic_geom_stacks}
    Let $X \in \dSt( \CAlg_{\bZp}, \tau_\et, \rmP_\sm)$ be a geometric stack with respect to the (derived) algebraic geometric context
        \[
            (\CAlg_{\bZp}, \tau_\et, \rmP_\sm).  
        \] 
    Then the $p$-completion of $X$, $X^\wedge_p \in \Shv(\adCAlg, \tau_\et)$,
    is geometric, i.e., $X^\wedge_p \in \adSt$.
\end{lemma}

\begin{proof} Suppose $X$ is $n$-geometric, for a given integer $n \geq 0$. Let
        \[
                \pi \colon P \to X  ,
        \]
    denote a smooth $(n-1)$-representable covering of $X$. We shall prove that the \emph{$p$-adic completion}
    \[
            (\pi)^\wedge_p \colon P^\wedge_p \to X^\wedge_p    
    \] 
    is itself a $(n-1)$-representable morphism with respect to the geometric context $(\adCAlg, \tau_\et, \rmP_\sm)$. Consider the pullback square
        \[
        \begin{tikzcd}
                Y \ar{r} \ar{d} & P^\wedge_p \ar{d}{(\pi)^\wedge_p} \\
                \Spf A \ar{r} & X^\wedge_p,
        \end{tikzcd}
        \]
    computed in the \infcat $\Shv(\adCAlg, \tau_\et)$. By construction, $Y$ can be realized as a limit diagram of pullback diagrams of the form
        \[
        \begin{tikzcd} 
            Y_n \ar{r} \ar{d} & P_n \ar{d}{\pi_n} \\
            \Spec A_n \ar{r} & X_n,
        \end{tikzcd}    
        \]
    computed in the \infcat $\dSt(\CAlg_{\bZp}, \tau_\et, \rmP_\sm)$, where the subscript $(-)_n$ denotes the reduction modulo the ideal $(p^n) \subseteq \bZp$. As $\pi_m$ is $(n-1)$-representable, for each $m \geq 1$, it follows that each
    $Y_m \in \Shv(\CAlg_{\bZp}, \tau_\et)$ is $(n-1)$-represntable. Therefore, we are reduced to show that the filtered colimit
        \[
            Y \simeq \underset{m \geq 1}{\colim} Y_m  
        \]
    is $(n-1)$-representable. As filtered colimits are preserved under finite limites we reduce ourselves, by induction on the geometric level, to the case $n= 0$. When $n=0$, the result follows from the universal property of the formal spectrum,
    proved in \cite[Proposition 8.1.2.1]{lurie2016spectral}.
\end{proof}

\begin{corollary} \label{cor_adic_geom_stacks}
    Let $\{X_n \}_{n \geq 1}$ denote a compatible \emph{ind-system} in the \infcat $\dSt(\CAlg_{\bZp}, \tau_\et, \rmP_\sm)$, where for each $n \geq 1$,
    $X_n$ denotes a derived $\bZ/p^n$-geometric stack. Then the induced functor
    \[
        \colim_{n \geq 1} X_n \colon \adCAlg \to \cS  
    \]
is naturally a derived $\bZp$-adic geometric stack.
\end{corollary}

\begin{proof}
    The proof follows exactly the same lines as the proof of \cref{const_of_adic_geom_stacks}.
\end{proof}

\begin{example}
    Both \cref{const_of_adic_geom_stacks} and \cref{cor_adic_geom_stacks} allow us to construct many interesting derived $\bZp$-adic geometric stacks. Indeed,
    let $X \in \dSch$ denote a derived $\bZp$-scheme, then its $p$-completion, $X^\wedge_p$ is naturally an object living the \infcat $\adSt$. 

    Given a $\bE_\infty$-group object in the \infcat $G \in \dSt(\CAlg_{\bZp}, \tau_\et, \rmP_\sm)$, the classifying space of the corresponding $p$-completion $\rmB G^\wedge_p \in \adSt$. Similarly, one can also consider classifying spaces of formal $\bZp$-formal groups
    as derived $\bZp$-adic geometric stacks.
    
    The $p$-completion of the derived moduli stack parametrizing perfect complexes, $\mathrm{Perf}$ is also naturally an object in the \infcat $\adSt$. So it is, the (derived) moduli stack classifying formal groups or $p$-divisible groups in mixed characteristic.
    Given any formal reductive group $G$, $\mathrm{Bun}_G$ also lives naturally in the \infcat $\adSt$.
\end{example}

\begin{remark}
    Let $X \in \St(\CAlg_{\bZp}, \tau_\et, \rmP_\sm)$. Thanks to Artin-Lurie representability theorem, \cite[Theorem 3.1.2]{lurie2012dag}, the functor of points associated to $X$ is nilcomplete, infinitesimally cartesian
    and it admits a global algebraic cotangent complex, which is an almost perfect complext on $X$. From the definitions, it is clear that $X^\wedge_p$ also possedes these properties, namely its corresponding functor of points is nilcomplete, infinitesimally cartesian and it admits a global almost perfect adic cotangent complex. Moreover,
    one has a natural equivalence
        \[
                \adL_{X^\wedge_p} \simeq (\bL^\alg_X  )^\wedge_p \in \Mod_{X^\wedge_p},
        \]
    where $\bL^\alg_X$ denotes the (algebraic) cotangent complex of $X \in \dSt(\CAlg_{\bZp}, \tau_\et, \rmP_\sm)$.
\end{remark}

\personal{One would like to have a Representability theorem in the context of derived $\bZp$-adic geometry. Otherwise, it will be difficult to state precisely that geometric stacks like $\Perf$ satisfy geometricity. Another way around this problem, might be by identifying such moduli
with completion along the ideal $(p) \subseteq \bZp$.}
\personal{Notice that Artin-Lurie representability holds true for Noetherian $\bE_\infty$-rings such that $\pi_0(R)$ is a Grothendieck ring. In particular, $\bZp$ is an example of such so we do have Artin-Lurie Representability theorem for \emph{algebraic} derived geometric $\bZp$-stacks.}


\personal{Notice that we would like that our statements are true more generally. In particular, we would like to be able to treat the case of the moduli of $p$-divisible groups.
However, this object is not geometric in our sense, since it does not admit a smooth covering. The only obstruction, it seems at this point, is to prove the existence of the local section, since for this we admit that we have
a smooth covering of the form $P \to Y$.}

\subsection{Brief considerations on fpqc-descent} \todo{Maybe it is more suitable to put this section in appendix.}

\begin{definition}
    Let $X \colon \adCAlg \to \cS$. We say that $X$ satisfies \emph{faithfully flat-descent} if for every faithfully flat morphism
        \[
           f \colon  A \to B  ,
        \]
    in the \infcat $\adCAlg$, one has a canonical equivalence
        \[
            X(\vC(B/A)) \to  X(A) 
        \]
    in the \infcat $\cS$, where $\vC(B/A)$ denotes the \emph{$p$-complete $\vC$ech nerve} associated to the morphism $f \colon A \to B$.
    \todo{Define the $p$-complete Cech nerve.}
\end{definition}

\begin{remark}
    Let $A \in \adCAlg$ and let $I \subseteq \pi_0(A)$ the defining ideal for $A$. Since we can identify (naturally) the underlying \inftopos of $Spec (A/I)$
    with the underlying \inftopos of $\Spf A$ we obtain that for each $A \in \adCAlg$, $\Spf A \in \dfSch_{\bZp}$ is quasi-compact.
\end{remark}

\begin{definition}
    Let $X \in \dfSch_{\bZp}$. An \emph{fpqc-covering} of $X$ is a family of morphisms $\{ f_i \colon X_i \to X \}$ of derived $\bZp$-adic schemes such that each $f_i$ is flat and such that for every affine open $U \subseteq T$ there exists an $n \geq 0$,
    a map $\alpha \colon \{1, \dots, n \}$ and affine opens $V_i \subseteq X_{\alpha(i)}$, $j=1, \dots, n$ with $\bigcap_{j=1}^n f_{\alpha(i)}(V_j) = U$.
\end{definition}

\begin{definition}
    Let $Y \in \mathrm{PShv}(\adCAlg)$ denote a pre-sheaf on $\adCAlg$. We say that $Y$ satisfies (hyper)-fpqc-descent if for every $\sfX \in \dfSch_{\bZp}$ together with a simplicial object
        \[
            U_\bullet \colon \rmN(\Delta^\op) \to \dfSch_{\bZp} , 
        \]
    such that each transition map is flat and quasi-compact,
    together with a canonical map $U_\bullet \to X$ wich is surjective, then the natural map
        \[
                Y(X ) \to Y(U_\bullet)  
        \]
    is an equivalence in the \infcat $\mathrm{PShv}(\adCAlg)$.
\end{definition}

\begin{lemma}
    Let $X \in \Shv(\CAlg_{\bZp}, \tau_\et)$ denote an \'etale sheaf with respect to the site $(\adCAlg, \tau_\et)$. Assume further that $X$ satisfies \emph{fpqc}-descent. Then the $p$-completion
        \[
                X^\wedge_p \colon \adCAlg \to \cS  
        \]
    also satisfies $\fpqc$-descent.
\end{lemma}

\begin{proof}
    The result is an immediate consequence of the fact that limits preserve limits. \todo{expand this proof.}
\end{proof}

\begin{proposition}
    The \infcat of fpqc-sheaves on $\adCAlg$, denoted $\Shv(\adCAlg, \tau_\fpqc)$ is subcanonical.
\end{proposition}

\begin{proof}
    This is an immediate consequence of \cite[Theorem 5.15]{Lurie_Spectral_Schemes} together with \cite[Proposition 8.1.2.1]{lurie2016spectral}.
\end{proof}
\subsection{Integral perfectoid algebras}

Let $R \in \adCAlg$ denote a (discrete) $\bZp$-algebra. We assume further that $R$ is equipped with an uniformizer $\pi \in R$ such that $R$ is
$\pi$-adically complete and separated. Denote by
    \[
        \varphi \colon R/ pR \to R/p  R
    \]
the \emph{absolute Frobenius} of the reduction $R/pR$. We define the \emph{tilt} of $R$, which we shall denote by $R^\flat$, as the (usual) inverse limit
    \[
        R^\flat \coloneqq \underset{\varphi }{\lim} R/ pR, 
    \]
in the category of (discrete) $\bF_p$-algebras. By construction, the tilt $R^\flat$ is a \emph{perfect $\bF_p$-algebra.}

\begin{definition}
    Let $R \in \adCAlg$, as above. We define \emph{Fontaine's ring} associated to $R$ as
        \[
                \Ainf(R) \coloneqq W(R^\flat).
        \]
    \personal{Maybe one can define $\Ainf(R)$ already as the pro-completion over the kernel of the projection map $\Ainf(R) \to R$.}
\end{definition}

\begin{remark}
    Let $R \in \adCAlg$ as above. Since $R^\flat$ is a perfect $\bF_p$-algebra, it follows that, for each integer $n \geq 1$, 
        \[
            \Ainf(R)/ p^n \simeq W_n(R^\flat).
        \]
    For this reason, the natural morphism
        \[
            R^\flat \to R/pR,  
        \]
    induced by projection on the first factor, induces (by the universal property of $p$-typical Witt vectors) a canonical morphism
        \[
                \theta \colon \Ainf(R) \to R,
        \]  
    in the \infcat $\adCAlg$.
\end{remark}

\begin{definition}
    Let $R \in \adCAlg$ be a discrete $\bZp$-adic algebra. We say that $R$ is an \emph{integral perfectoid $\bZp$-adic algebra} if if satisfies the following conditions:
        \begin{enumerate}
                \item There exists an element $\pi \in R$ such that $R$ is $\pi$-adically complete and separated. We assume further that $\pi^p \vert \ p$;
                \item The Frobenius morphism    
                        \[
                                \varphi \colon R/p R \to R/ pR
                        \]  
                  is surjective;
            \item The kernel of the canonical morphism $\theta \colon \Ainf(R) \to R$ is generated by a single element.
        \end{enumerate}
\end{definition}

\begin{example} The following is a transcription of \cite[Example 3.6]{bhatt2018integral}.
    \begin{enumerate}
        \item Let $R \in \adCAlg$ be a discrete $\bZp$-adic algebra. If $pR =0 $ then $R$ is integral perfectoid if and only if $R$ is perfect as a $\bF_p$-algebra.
        \item Let $\cycbZp \coloneqq (\bZp[\zeta_{p^\infty}])^\wedge_p$ denote the $p$-adic completion of the ring of integers of the cyclotomic extension $\bQ_p(\zeta_{p^\infty}) / \bQ_p$. Then $\bZp^\cyc$ is a perfectoid $\bZp$-algebra.
        \item The $p$-adic completion of the ring of integers of any algebraic extension of $\bQ_p(\zeta_{p^\infty})$ is an integral perfectoid $\bZp$-adic algebra.
        \item The $p$-adic complete ring $\bZp^\cyc \langle T^{1/p^\infty} \rangle$ is itself a perfectoid $\bZp$-algebra. So it is the \emph{perfectoid torus}, $\bZp^\cyc \langle T^{\pm 1/p^\infty} \rangle$.
    \end{enumerate}
\end{example}

\begin{lemma}
    Let $R \in \adCAlg$ be an integral perfect $\bZp$-adic algebra. Then $R$ is (derived) $p$-complete.
\end{lemma}

\begin{proof}
    Let $R \in \adCAlg$ be a discrete $\bZp$-adic algebra. We wish to show that $R$ is $p$-complete provided that $R$ is $p$-torsion free (in the usual sense). Thanks to \cite[Lemma 8.1.2.3]{lurie2016spectral}, it suffices to show that the canonical map
        \[
            R \to \underset{n \geq 1}{\lim} R_n
        \]  
    is an equivalence in the \infcat $\adCAlg$. In our situation, we can identify, for each $n \geq 1$, 
        \[
            R_n \simeq R \otimes_{\bZp} \bZ/ p^n\bZ .
        \]
    Therefore, for each integer $i \geq 0$, $\pi_i(R_n) = 0$, except in the case where $i=0, \ 1$, in which we have $\pi_1(R_n) = R[p^n] =0$ and $\pi_0(R_n) = R/p^n$. Passing to inverse limits, the Milnor short exact sequence implies that $\pi_i(R^\wedge_p) =0$ for $i>1$. For $i=0$, we have a Milnor short exact sequence of the form
        \begin{equation} \label{p-tors_perf}
            0 \to \underset{n \geq 1}{\lim^1} \ \pi_1(R_n) \to \pi_0(R^\wedge_p) \to \underset{n \geq 1}{\lim} \ \pi_0(R_n)= R \to 0,
        \end{equation}
    and for $i=1,$ we have an equivalence $\pi_1(\lim R_n) \simeq \underset{n \geq 1}{\lim^1} \pi_1(R_n)$. Therefore, in order to prove the claim it suffices to show that the pro-system $\{ \pi_1(R_n) \}_{n \geq 1}$ is trivial. Every integral perfectoid $\bZp$-algebra
    has bounded $p$-torsion, \cite[\todo{Add reference}]{bhatt2018topological}. Therefore, for each $m > 0$ there exists a sufficiently large integer $n > m$ such that the transition map $R[p^n] \to R[p^{m}]$ vanishes. The result then follows, as desired.
\end{proof}

\begin{corollary}
    Let $R \in \adCAlg$ denote an integral perfectoid $\bZp$-algebra. Then $\Ainf(R) \in \adCAlg$ is $p$-adically complete and separated and $p$-torsion free. In particular, $\Ainf(R)$ is
    (derived) $p$-complete.
\end{corollary}

\begin{proof}
    By construction, we have that $\Ainf(R) = W(R^\flat)$. The (discrete) $\bF_p$-algebra $R^\flat$ is a perfect $\bF_p$-algebra. For this reason, $W(R^\flat)$ is $p$-adically complete, separated and $p$-torsion free.
\end{proof}

\begin{remark}
    We have a natural isomorphism $\Ainf(R) / p \simeq R^\flat$ of (discrete) $\bZp$-adic algebras.
\end{remark}

\begin{lemma} \label{pro_thick}
    The morphism $\theta \colon \Ainf(R) \to R$ exhibits the ring $\Ainf(R)$ as the universal pro-thickening of $R$.
\end{lemma}

\begin{proof}
    This is the content of \cite[Proposition 3.14]{szamuely2016p}. \todo{The reference is not quite right since it is less general, even if the proof applies for any integral perfectoid. Also better to use the original Fontaine reference!}
\end{proof}

The following result is of fundamental importance for our purposes:

\begin{proposition} \label{prop_non_obs_perf}
    Let $R \to S$ be a morphism between integral perfectoid $\bZp$-adic algebras in the \infcat $\adCAlg$. Then $\adL_{S/R} \simeq 0$.
\end{proposition}

\begin{proof}
    This is the content of \cite[Proposition 3.14]{bhatt2018integral}.
\end{proof}

\begin{remark}
    Let $f \colon R \to S$ be a morphism between integral perfectoid $\bZp$-algebras in the \infcat $\adCAlg$. The morphism $f$ induces a morphism, at the level of the tilts,
        \[
            (f)^\flat \colon R^\flat \to S^\flat  
        \]
    which is \emph{formally \'etale}. By applying the $p$-typical Witt vectors construction one obtains an induced morphism at the level of Fontaine's rings
        \[
            \widetilde{f} \colon \Ainf(R) \to \Ainf(S). 
        \]
    Thanks to \cref{pro_thick} above,
    \cite[Proposition 8.4.2.5, see also Remark 8.4.2.3]{Lurie_Higher_algebra} together with formal \'etaleness of $f$ combined with the fact that 
        \[
            S \simeq  \Ainf(S) \otimes_{\Ainf(R)} R \in \adCAlg,
        \]
    that $\widetilde{f}$ is the unique (up to contractible indeterminacy) such deformation of $f$, in the \infcat $\adCAlg$.
\end{remark}


The following two results will establish the precise relation between $\Ainf(R) $ and $R$, via obstruction theory, for perfectoid $R$:



\begin{proposition} \label{Ainf_de_Rham}
    Let $R \in \adCAlg$ be a $p$-torsion free integral perfectoid $\bZp$-algebra. Then, for every $p$-complete $A \in \adCAlg$, the natural morphism
        \[
               \theta \colon \Map_{\adCAlg}(\Ainf(R), A) \to   \Map_{\adCAlg}(R, \pi_0(A)_\mathrm{red}) ,
        \]
    is an equivalence of mapping spaces.
\end{proposition}

\begin{proof}
    Let $f \colon R \to \pi_0(A)_\mathrm{red}$ denote a continuous $p$-adic morphism. Given any continuous morphism $f \colon R \to \pi_0(A)_\mathrm{red}$ we obtain, by base change along the morphism $\bZp \to \bF_p$, a well defined morphism
        \[
            f \otimes_{\bZp} \bF_p \colon R/p \to \pi_0(A)_{\mathrm{red}} \otimes_{\bZp} \bF_p . 
        \]
    Pre-composition with the natural map $R^\flat \to R/p$ induces thus a morphism
        \[
            \overline{f}_1 \colon R^\flat \to \pi_0(A)_{\red} \otimes_{\bZp} \bF_p.  
        \]
    Since $R^\flat $ is a perfect $\bF_p$-algebra, it follows that the morphism $\overline{f}_1$ factors (uniquely) through the perfectization of  $\pi_0(A)_{\red} \otimes_{\bZp} \bF_p$,
    denoted 
        \[
            ( \pi_0(A)_\red \otimes_{\bZp} \bF_p)^{\mathrm{perf}} \coloneqq \underset{\varphi}{\lim} \ \big( \pi_0(A)_\red \otimes_{\bZp} \bF_p \big).
        \]
    Thanks to \cite[Proposition 11.6]{bhatt2017projectivity} it follows that $( \pi_0(A)_\red \otimes_{\bZp} \bF_p)^{\mathrm{perf}}$ is a discrete $\bF_p$-algebra and we have a chain of uniquely defined (up to contractible indeterminacy) chain of equivalences
        \begin{align}
            ( \pi_0(A)_\red \otimes_{\bZp} \bF_p)^{\mathrm{perf}} & \simeq (\pi_0(A) \otimes_{\bZp} \bF_p)_\red^\mathrm{perf} \\
                                                                  &   \simeq \pi_0(A \otimes_{\bZp} \bF_p)_{\red}^\mathrm{perf} \\
                                                                  &    \simeq \pi_0(A \otimes_{\bZp} \bF_p)^{\mathrm{perf}}.
        \end{align}
    The natural morphism $\pi_0(A \otimes_{\bZp} \bF_p)^{\mathrm{perf}} \to \pi_0(A \otimes_{\bZp} \bF_p)$ induces thus a canonical morphism
        \[
                \mathfrak{f}_0(1) \colon R^\flat \to \pi_0(A \otimes_{\bZp} \bF_p).
        \]  
    We shall prove, by induction on the (adic) Postnikov tower for $A \in \adCAlg$, that $\mathfrak{f}_0(1)$ lifts uniquely (up to a contractible space of indeterminacy) to a well defined morphism
        \[
            \mathfrak{f}(1) \colon R^\flat \to A \otimes_{\bZp} \bF_p.
        \]
    The case $n=0$ has already been dealt with. Suppose now that for $n \geq 0$ we have constructed a unique, up to contractible indeterminacy, morphism
        \[
            \mathfrak{f}_n(1) \colon R^\flat \to \tau_{\leq n} ( A \otimes_{\bZp} \bF_p)  
        \]
    in the \infcat $\adCAlg$. Indeed, consider the \emph{$n+1$-th level (Postnikov) derivation for $A \otimes_{\bZp} \bF_p$},
        \[
                d_n \colon \tau_{\leq n} (A \otimes_{\bZp} \bF_p) \to \tau_{\leq n}(A \otimes_{\bZp} \bF_p) \oplus \pi_{n+1}(A \otimes_{\bZp} \bF_p)[n+2].
        \]
    The derivation $d_n$ fits into a pullback square
        \[
        \begin{tikzcd}
                \tau_{\leq n+1} (A \otimes_{\bZp} \bF_p) \ar{r} \ar{d} & \tau_{\leq n}(A \otimes_{\bZp} \bF_p)   \ar{d} \\
                \tau_{\leq n} (A \otimes_{\bZp} \bF_p) \ar{r} & \tau_{\leq n}(A \otimes_{\bZp} \bF_p) \oplus \pi_{n+1}(A \otimes_{\bZp} \bF_p)[n+2]
        \end{tikzcd}
        \]
    in the \infcat $\adCAlg$. Moreover, liftings of the morphism 
        \[
            \mathfrak{f}_n(1) \colon R^\flat \to \tau_{\leq n}(A \otimes_{\bZp} \bF_p)
        \] 
    along the (adic) square-zero extension $\tau_{\leq n+1} (A \otimes_{\bZp} \bF_p ) \to \tau_{\leq n} (A \otimes_{\bZp} \bF_p)$ are classified by the mapping space
        \[
                \Map_{\Mod_{R^\flat}} \bigg(\adL_{R^\flat}, \pi_{n+1}(A \otimes_{\bZp} \bF_p) [n+2] \bigg) \simeq * \in \cS,
        \]
    since, as $R^\flat$ is a perfect $\bF_p$-algebra, one necessarily has $\adL_{R^\flat} \simeq 0$.
    Therefore, the morphism $\mathfrak{f}_1(n)$ lifts into a uniquely (up to contractible indeterminacy) well defined morphism
        \[
                \mathfrak{f}_1(n+1) \colon R^\flat \to \tau_{\leq n+1}(A \otimes_{\bZp} \bF_p),  
        \]
    in the \infcat $\adCAlg$. This concludes the inductive step. Passing to inverse limits we obtain a well defined morphism
        \[
                \mathfrak{f}_1 \colon R^\flat \to A \otimes_{\bZp} \bF_p
        \]
    in the \infcat $\adCAlg$.
    Moreover, as $R^\flat$ is a perfect $\bF_p$-algebra, it follows that $\Ainf(R)/p^n \simeq W_n(R/p^\flat)$ in the \infcat $\adCAlg$. By the universal property of $p$-typical Witt vectors one obtains, for each integer $n \geq 1$, uniquely defined (up to contractible indeterminacy) morphisms
        \[
            \mathfrak{f}_n \colon \Ainf(R)/p^n \to A_n.
        \]  
    \personal{Notice that the universal property of Witt vectors is derived in nature. For example, given two objects $A, \ B \in \adCAlg$ we can find resolutions by polynomial algebras of these, seen as $\bZp$-simplicial algebras.
    We obtain thus a morphism of simplicial polynomial algebras of the form $ \varphi \colon P_\bullet \to Q_\bullet$. If $Q_\bullet$ is defined over some $\bZ_p/p^n$ then applying Witt vectors component-wise to $P_\bullet$ provides a uniquely defined (up to contractible homotopy) $W_n(P_\bullet) \to Q_\bullet$ and thus $W_n(A) \to B$ in $\adCAlg$.}
    Passing to the limit one obtains a unique lift (up to contractible indeterminacy) $\mathfrak{f} \colon \Ainf(R) \to A$, in $\adCAlg$.
    Therefore, we conclude that the morphism
        \[
            \theta \colon \Map_{\adCAlg}(\Ainf(R), A) \to \Map_{\adCAlg} (R, \pi_0(A)_\red)  
        \]
    has contractible fiber over any $f \in \Map_{\adCAlg} (R, \pi_0(A)_\red)$. We conclude that the morphism $\theta$ is an equivalence of spaces, as desired.
\end{proof}


\begin{corollary}
    Let $R \in \adCAlg$ be an integral perfectoid $\bZp$-algebra. Then the canonical morphism
        \[
            \Spf R \to \Spf(\Ainf(R)),
        \]
    where we consider $\Ainf(R) \in \adCAlg$ equipped with its $(p, \varepsilon)$-topology, exhibits the later as the \emph{de Rham} stack of $\Spf R$, in the \infcat $\Shv(\adCAlg, \tau_\et)$.
\end{corollary}

\begin{proof}
    This is a direct consequence of \cref{Ainf_de_Rham} and the construction of the de Rham stack associated to a formal spectrum.
\end{proof}

\begin{proposition}
Consider a pullback diagram of the form
    \[
    \begin{tikzcd}
            \Spf(S) \ar{r} \ar{d}{f} & \tP \ar{d}{\widetilde{f}} \\
            \Spf(R) \ar{r} & \Spf (\Ainf(R))
    \end{tikzcd}
    \]
in the \infcat $\dfSch_{\bZp}$, where $S$ and $R$ are $p$-torsion free integral perfectoid and we consider $\Ainf(R)$ with its natural $(p, \varepsilon)$-adic topology. Then $\tP \simeq \Spf (\Ainf(S))$. 
\end{proposition}

\begin{proof} Notice first that $\tP$ is a thickening of $\Spf(S)$ and the morphism $\tP \to \Spf(\Ainf(R))$ is a deformation of the morphism $\Spf(S) \to \Spf(R)$.
    Thanks to \cite[\href{https://stacks.math.columbia.edu/tag/06AD}{Tag 06AD}]{stacks-project} it follows that $\tP$ is affine. Therefore we can write
        \[
            \tP \simeq \Spf A,
        \]
    where $A$ is a simplicial $\bZp$-adic algebra. \cref{prop_non_obs_perf} implies that $\adL_{S/ R} \simeq 0$. Moreover, for every $n \geq 1$, \cite[Theorem 8.4.2.7]{Lurie_Higher_algebra} implies that any deformation of the morphism $f$, over $\Spf(\Ainf(R)/\varepsilon^n)$, is controlled by the mapping space
        \[
                \Map_{\Mod_S} \big(\adL_{S/R}, S \otimes_R ( \varepsilon)/(\varepsilon^n) \big) \simeq 0.
        \]  
    Therefore, such deformation must be unique (up to contractible indeterminacy). Since the morphism
        \[
            \Ainf(S)/(\varepsilon_S^n) \to \Ainf(R)/(\varepsilon^n)
        \]
    is such a deformation, we conclude by passage to the limit that we have a canonical equivalence
        \[
                \tP \simeq \Spf(\Ainf(S))  
        \]
    and the morphism $\widetilde{f} \colon \tP \to \Spf(\Ainf(R))$ coincides with the morphism induced by the corresponding map
        \[
                \Ainf(R) \to \Ainf(S),
        \]
    (uniquely) induced from $R \to S$.
\end{proof}


\subsection{de Rham stack}

\begin{definition}
    Let $X \in \Shv(\adCAlg, \tau_\et)$ denote an \'etale sheaf on $\adCAlg$. Then, we defined its associated \emph{de Rham sheaf} 
        \[
            X_\dR \colon \adCAlg \to \cS
        \]  
    given on objects by the formula
        \[
            A \in \adCAlg \mapsto X(\pi_0(A)_\red) \in \cS. 
        \]
\end{definition}

\begin{lemma}
    Let $X \in \Shv(\adCAlg, \tau_\et)$ and consider its associated de Rham sheaf $X_\dR \in \Shv(\adCAlg, \tau_\et)$. If $X$ satisfies fpqc-descent then so does $X_\dR$.
\end{lemma}

\begin{proof}
    One is reduced to show that the association $A \in \adCAlg \mapsto A_\red \in \adCAlg$ satisfies faithfully flat descent. This statement should reduce itself to prove that Jacobson ideal satisfies faithfully flat descent. This last assertion is a special case of fpqc-descent for quasi-coherent sheaves, proved in \cite[\href{https://stacks.math.columbia.edu/tag/023R}{Tag 023R}]{stacks-project}. 
\end{proof}

\subsection{Faithfully flat coverings by perfectoids}
The following result shows that every object $A \in \adCAlg$ which is regular and Noetherian admits a faithfully flat map to a perfectoid ring:

\begin{theorem}
    Let $A \in \adCAlg$ be a discrete $\bZp$-adic algebra. Then $A$ is Noetherian and regular if and only if exists a faithfully flat morphism
        \[
                A \to A_\infty  ,
        \]
    in the \infcat $\adCAlg$, where $A_\infty \in \adCAlg$ is an integral perfectoid $\bZp$-algebra.
\end{theorem}

\begin{proof}
    This is precisely the content of the main result \cite[Theorem 4.7]{bhatt2019regular}.
\end{proof}

\begin{remark}
    Let $\bT^n \coloneqq \Spf (\bZp \langle T_1^{\pm 1}, \dots , T_n^{\pm 1} \rangle)$ denote the formal torus of dimension $n$. We define
        \[
                \bT^n(p^\infty) \coloneqq \Spf ( \cycbZp \langle T_1^{\pm 1/p^\infty}, \dots, T_n^{\pm 1/p^\infty})  .
        \]
    We have a canonical morphism
        \[
            \pi_n \colon \bT^n(p^\infty) \to \bT^n  
        \]
    which is an fpqc-covering and exhibits $\bT^n(p^\infty)$ as a $\Gamma \coloneqq \bZp(1)^n$-torsor over $\bT^n$. \todo{justify this with a refenrece. Does $\Gamma$ really have the correct dimension?}
    In this case, we construct an explicit faithfully flat covering of $\bT^n$. Moreover, suppose that we have an \'etale morphism \todo{Does \'etale really suffices or do we really finite \'etale? Check this.}
        \[
                \Spf R \to \bT^n,
        \]
    where $ R \in \adCAlg$. Then we have a perfectoid covering
        \[
                \Spf (R_\infty) \to \Spf(R)  
        \]
    where $R_\infty \coloneqq R \otimes_{\bZp \langle T_1^{\pm 1}, \dots , T_n^{\pm 1} \rangle} \cycbZp \langle T_1^{\pm 1/p^\infty}, \dots, T_n^{\pm 1/p^\infty} \rangle$.
    Then $R_\infty$ is an integral perfectoid $\bZp$-algebra,\todo{Check this!} which provides us with a faithfully flat morphism
        \[
                R \to R_\infty    
        \] 
    in the \infcat $\adCAlg$. \todo{Check that $\Spf R_\infty$ is quasi-compact.}
\end{remark}

\subsection{Smooth morphisms between derived $\bZp$-adic geometric stacks}

In this \S we wish to prove the following:

\begin{theorem}
    Let $f \colon X \to Y$ denote a smooth morphism in the \infcat $\adSt$. Then, locally on both $X$ and $Y$, $f$ can be factored as a composite
        \[
        \begin{tikzcd}
                X \ar{r}{g} & Y \times \fA^n_{\bZp} \ar{r}{\mathrm{pr}_1} & Y
        \end{tikzcd}
        \]
    where $g$ is an \'etale map of derived $\bZp$-adic geometric stacks and $\pr$ denotes the canonical projection.
\end{theorem}

\begin{proof}
    Since the result is both local on $X$ and $Y$, we can assume that $f \colon X \to Y$ is a smooth morphism between derived $\bZp$-adic schemes, \todo{Check this!}. In this case, the proof of \cite[Proposition 5.50]{porta2017representability} applies. \todo{Check this and write it better!}
\end{proof}

\subsection{Assumptions}

First of all we will define the context in which we are going to work through.  Let us denote $\adCAlg$ the \infcat of adic $\bZp$-simplicial algebras, introduced in \cite{antonio2018p}. We consider the \infcat $\adSt(\adCAlg, \tau_{\et}, \rmP_{\sm})$ of \emph{derived $\bZp$-adic geometric stacks}.
Suppose we are given a diagram of the form
    \[
    \begin{tikzcd}
        M \ar{r}{p} \ar{d}{q} & X \\
        Y 
    \end{tikzcd}
    \]
where $M \in \dfSch$ is a smooth $\bZp$-adic scheme and $X, \ Y \in \adSt(\adCAlg, \tau_{\et}, \rmP_{\sm})$. Throughout the text, $p$ denotes a smooth surjective morphism and $q$ a smooth morphism of derived $\bZp$-adic geometric stacks. We assume further that $Y$ satisfies \emph{fpqc descent}.
\personal{The need to assume that $Y$ satisfies fpqc descent boils down to the fact that the \'etale topology is not sufficient for many of our purposes. Namely, in order to compute the correct mapping spaces of derived $\bZp$-adic geometric stacks one needs to pass to perfectoid coverings, in a similar vein as in BMS1. 
More precisely, given a Noetherian regular algebra $\bZp$-algebra $A$, by a theorem of Bhatt, Iyegar and Ma one knows that there exists a faithfully flat morphism
$A \to A_\infty$, where $A_\infty$ is integral perfectoid. Furthermore, by a celebrated result of Abhyankar one can suppose further that $A_\infty$ admits all $p$-th power roots of unity, thus living over the perfectoid covering $\cycbZp$ of $\bZp$. However, it seems that, in general, the morphism $A \to A_\infty$ is not \'etale or even weak \'etale in the sense of \cite{bhatt2013pro}.
Therefore, in order to be able to reduce our local computations to computations involving the perfectoid nature of suitable algebras, one needs to assume $Y$ to satisfy fpqc descent.
This is indeed the case when $Y = \Perf, \ \mathrm{QCoh}^\heartsuit, \ \mathrm{QCoh}, \mathrm{Bun}_G$, where $G$ denotes a reductive group, $X$ a formal scheme, $\mathrm{B} G$, for $G$ a formal group scheme or a $p$-divisible group, $\mathcal{M}_{\mathrm{FM}}$ the moduli space of $p$-divisible groups, etc.}

Our goal is to prove a mixed characteristic analogue of a theorem of Van Est, in the context of (derived) differential geometry, proved in great generality by J. Nuiten using the Koszul duality for Lie algebroids, see \cite{nuiten2017koszul}. 
\todo{Recall the statement}
\todo{Check that the geometricity of the stack $Y$ is really need or we can relax a bit the assumptions on $Y$, in order for the result to be compatible with $\cM_{\FormalModels}$, for example.}
\section{The existence of a local section}

We want to prove a connectivity statement for the canonical morphism of mapping spaces
    \[  
        \Map_{M/ }\big( X, Y \big) \to \Map_{\adSt} \big( X^\wedge_M , Y^\wedge_M \big) .
    \]

Suppose then that we are given a morphism of sheaves
    \[
        X^\wedge_M \to Y  .
    \]
Assume further that we have a smooth covering $\pi \colon P \to Y $, in the \infcat $\adSt(\adCAlg, \tau_{\text{\'et}}, \rmP_{\sm})$ and form the pullback square
    \[
    \begin{tikzcd}
            P' \ar{r} \ar{d} & P \ar{d} \\
            X^\wedge_M \ar{r}  & Y.
    \end{tikzcd}
    \]
In particular, the natural morphism $P' \to X^\wedge_M$ is a smooth covering. Moreover, we have a natural inclusion morphism   
    \[
        M \to X^\wedge_M.
    \]
Our goal is to lift the morphism $X^\wedge_M \to Y$ to a morphism $M \to P$ which induces a well defined morphism at the level of \v{C}ech nerves 
    \[
        \vC(M_\bullet / X) \to \vC(P_\bullet / Y). 
    \]
In order to do so, we will construct a section of the smooth morphism 
    \[
        P' \to X^\wedge_M,  
    \]
which induces, via composition, a morphism $M \to P$. The construction of the section is a local construction and it is precisely in this situation that we will use the setting of integral perfectoid $\bZp$-algebras.

\begin{lemma}
If such a section to $X^\wedge_M \to P'$ exists then it induces such a morphism at the level of \v{C}ech nerves
    \[
        \vC(M_\bullet / X) \to \vC(P_\bullet / Y)
    \]
\end{lemma}

\subsection{First reduction step} Let $r \colon U \to X$ denote a morphism in $\adSt( \adCAlg, \tau_{\'et}, \rmP_\sm)$ such that
    \[
        U \coloneqq \Spf A  
    \]
is a derived affined $\bZp$-adic scheme. 

\personal{Do we have to impose assumptions on the morphism $U \to X$, such as being Zariski open, \'etale?}

\begin{proposition} \label{stat_1} Up to shrienking $U$ we can suppose that the pullback of the smooth covering map $p \colon M \to X$ along $r$ can be realized as a composition of the form
    \[
    \begin{tikzcd}
        U \times \Spf R \ar{r}{g} & U \times \bT^n \ar{r}{\pr} & U ,
    \end{tikzcd}
    \]
where $g$ is an \'etale morphism. Moreover, $g$ itself factors as a composite
    \[
    \begin{tikzcd}
            U \times \Spf R \ar{r}{g'} & U \times W \ar{r}{s} & U \times \bT^n,
    \end{tikzcd}  
    \]
where $g'$ corresponds to a rational domain inclusion and $s$ is a finite \'etale map.
\end{proposition}

\begin{remark}
    The second part of the above statement might not be necessary for our purposes. However I am not sure yet.
\end{remark}

\begin{notation}
    We will denote by $\cycbZp$ the integral perfectoid $\bZp$-algebra obtained from $\bZp$ by adding all $p$-th roots of unity to $\bZp$ and performing a $p$-adic completion.
\end{notation}
Assuming \cref{stat_1} holds, one can further proceed to show that:


\begin{lemma}
    There exists a commutative diagram, \personal{possibly cartesian and fpqc coverings?}, of the form
    \begin{equation}
    \begin{tikzcd} \label{diag_chain_pulls}
            U^\cyc \times_{\Spf \cycbZp} \Spf R_\infty \ar{r} \ar{d} & U \times \Spf R \ar{d} \\  
            U^\cyc \times_{\Spf \cycbZp} \bT^n(p^\infty) \ar{r} \ar{rd} & U \times \bT^n \ar{d} \\
             & U
    \end{tikzcd}
    \end{equation}
    in the \infcat $\Shv_{\fpqc}(\adCAlg)$. Here 
        \[
            U^\cyc \coloneqq U \times \Spf \cycbZp
        \]
    and 
         \[
            \bT(p^\infty) \to \bT 
        \]
        denotes the perfectoid cover of the torus $\bT$, given by adding $p$-th roots of unity to $\bZp$ and $p$-th roots of the (free invertible) variables on $\bT$.
\end{lemma}

\begin{lemma}
    The \emph{fpqc}-covering $\bT(p^\infty) \to \bT$ is a $\Gamma \coloneqq \prod_n \bZp(1)$-torsor, for the \emph{fpqc}-topology, which is compatible with the \emph{fpqc}-$\bZp(1)$-torsor
        \[
                \Spf \cycbZp \to \Spf \bZp.  
        \]
\end{lemma}


\begin{proposition}
    The pullback square of derived $\bZp$-adic geometric stacks
        \[
        \begin{tikzcd} 
            U \times \Spf R \ar{r} \ar{d} & M \ar{d}{p} \\
            U \ar{r} & X
        \end{tikzcd}
        \]
    induces a canonical morphism $U \times ( \Spf R)_{\dR} \to X^\wedge_M$. Moreover, if the morphism $U \to X$ is surjective or a covering (namely, an effective epimorphism of sheaves) then
    so it is the canonical morphism
        \[
                U \times (\Spf R)_{\dR} \to X^\wedge_M.  
        \]
\end{proposition}

\subsection{Further reductions} Our initial situation can be thus translated as follows: consider the chain of pullback diagrams
    \[
    \begin{tikzcd}
        \tP'' \ar{r} \ar{d} & \tP' \ar{r} \ar{d} & \tP \ar{r} \ar{d} & P \ar{d} \\
        U \times \Spf R \ar{r} \ar{d} & M \ar{r} \ar{d} & X^\wedge_M \ar{r} & Y \\
        U \ar{r} & X & &
    \end{tikzcd}.
    \]
The induced morphism $U \times \Spf R \to X^\wedge_M$ factors as
    \begin{equation} \label{diag1}
            U \times \Spf R \to U \times (\Spf R)_{\dR} \to X^\wedge_M . 
    \end{equation}
Consider the following \emph{pullback} diagram in the \infcat, $\Shv_{\fpqc}(\adCAlg)$, \personal{is it really a pullback diagram?},
    \[
    \begin{tikzcd}
            \Spf R_\infty \ar{r} \ar{d}{\theta} & \Spf R \ar{d} \\ 
            \Spf \AdR (R_\infty) \ar{r} & (\Spf R)_\dR,
    \end{tikzcd}
    \]
where  $\AdR(R_\infty)$ denotes the completion with respect to the kernel (structural) ring homomorphism 
    \[
        \theta \colon \Ainf (R_\infty) \to R_\infty, 
    \]
which, by the perfectoid nature of $R_\infty$, is generated by a single elemet $\varepsilon \in \Ainf(R_\infty$.

\begin{lemma}
        Let $R_\infty \in \adCAlg$ denote a perfectoid $\bZp$-algebra and consider the (structural) morphism
            \[
                \theta \colon \Ainf(R_\infty) \to R_\infty  .
            \]
        Then $\theta$ exhibits $\Spf \AdR(R_\infty)$ as the \emph{de Rham stack} associated to $\Spf R_\infty$, $(\Spf R_\infty)_\dR$. Furthermore, if we are given a faithfully flat morphism $R \to R_\infty$, where $R$ is a regular Noetherian $\bZp$-adic algebra, then
        $\Spf \AdR(R_\infty) \to (\Spf R)_{\dR}$ is a fpqc morphism.
\end{lemma}

Restricting further along the faithfully flat morphisms $\Spf R_\infty \to \Spf R$ and $\Spf \AdR(R_\infty) \to (\Spf R)_{\dR}$ one obtains thus a pullback square of the form
    \begin{equation} \label{diag_local}
    \begin{tikzcd}
        P_\infty \ar{r} \ar{d} & \tP_\infty \ar{d} \\
        U^\cyc \times_{\Spf \cycbZp} \Spf R_\infty \ar{r} & U^\cyc \times_{\Spf \cycbZp} \Spf \AdR(R_\infty)
    \end{tikzcd}    
    \end{equation}
in the \infcat $\Shv_{\fpqc}(\adCAlg)$.

\begin{remark}
Notice that, up to shrienking $U$ mapping into $X$, one can assume that the smooth map $\tP'' \to U \times \Spf R$, displayed in diagram \eqref{diag_chain_pulls}, is actually of the form
    \[
            U \times \Spf R \times \Spf S \to U \times \Spf R ,  
    \]
where $\Spf S $ admits a suitable \'etale morphism to a torus $\Spf S \to \bT^m$. 
\end{remark}

For this reason, building upon what we have discussed so far, one can actually suppose that the diagram \eqref{diag_local} is of the form
    \[
    \begin{tikzcd}
        U^\cyc \times_{\Spf \cycbZp} \Spf S_\infty \times_{\Spf \bZp^\cyc} \Spf R_\infty \ar{r} \ar{d} & \tP \ar{d} \\
        U^\cyc \times_{\Spf \cycbZp} \Spf R_\infty \ar{r} & U^\cyc \times_{\Spf \bZp^\cyc} \Spf \AdR(R_\infty),
    \end{tikzcd}
    \]
where $S_\infty$ is a perfectoid $\bZp$-algebra which admits a faithfully flat morphism of the form $S \to S_\infty$. Moreover, as before, we can assume that $S_\infty$ admits all $p$-th roots of unity and $p$-th roots of the (invertible free) variable overs $\bT^m$ (\todo{make this assertion more precise}). In particular, we deduce that $S_\infty$ is naturally a $\cycbZp$-algebra.

\subsection{Existence and descent properties of local sections} In order to construct the desired section one would be reduced to show the following:

\begin{notation}
    We denote by $\cotimes$ the $p$-complete tensor product in the \infcat $\adCAlg$.
\end{notation} 

\begin{proposition}
    The \emph{adic cotangent complex} 
        \[
                \adL_{S_\infty \cotimes_p R_\infty / R_\infty} \simeq 0.  
        \]
    In particular, there are no obstructions to find a deformation of the morphism
        \[
                U^\cyc \times_{\Spf \cycbZp} \Spf S_\infty \times_{\Spf \cycbZp} \Spf R_\infty \to U^\cyc \times_{\Spf \cycbZp} \Spf R_\infty.  
        \]
\end{proposition}

\begin{corollary}
    The deformation morphism $\widetilde{q} \colon \tP \to U^\cyc \times_{\cycbZp} \Spf \AdR(R_\infty)$ of the projection
        \[
                U^\cyc \times_{\Spf \cycbZp} \Spf S_\infty \times_{\Spf \cycbZp} \Spf R_\infty \to U^\cyc \times_{\cycbZp} \Spf \AdR(R_\infty)  
        \]
    corresponds to the trivial deformation. In particular, it admits a section.
\end{corollary}

\begin{theorem}
    There exists a choice of a section $s \colon \Spf \AdR(R_\infty) \to \tP$ of $\widetilde{q}$ which is $\Gamma$-equivariant. In particular, it descends to a section of the morphism
        \[
            \widetilde{q} \colon \tP \to U \times (\Spf R)_{\dR}  
        \]
\end{theorem}

\personal{In order for the section $s$ constructed above to descend, does one need to show that 
    \[
        \widetilde{q} \colon \tP \to U^\cyc \times_{\cycbZp} \Spf \AdR(R_\infty)
    \]
    $\widetilde{q} \colon \tP \to U^\cyc \times_{\cycbZp} \Spf \AdR(R_\infty)$
is also a $\Gamma$-torsor?}


\bibliographystyle{plain}
\bibliography{Van_Est}
\end{document}