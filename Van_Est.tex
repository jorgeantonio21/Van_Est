\pdfoutput=1
%The other issue is that some packages, such as microtype, produce different output under pdflatex. By default the arXiv goes from dvi to ps to pdf, so if you need pdflatex you have to set the \pdfoutput flag in the TeX file.
\newif\ifpersonal
\personaltrue % comment to remove personal notes

\documentclass[10pt,a4paper]{amsart}
\usepackage{fullpage}
%\linespread{1.1}
%\usepackage{amsmath, amscd, amssymb, amsthm, latexsym, url, color, todonotes} %pdflscape}
%\usepackage{graphicx}
\usepackage{mathrsfs}
\usepackage[left=1.0in, right=1.0in, top=1in, bottom=1.3in, includefoot, headheight=13.6pt]{geometry}
\usepackage[colorlinks=true,hyperindex,citecolor=blue,linkcolor=black]{hyperref}
\input{xy}
%\cXyoption{all}
%\usepackage{natbib}
\usepackage{cite}
\usepackage{tikz-cd}
%\usepackage{stix}
\setlength{\marginparwidth}{1in}
\usepackage[capitalize]{cleveref}
%\usepackage[utf8]{inputenc}
\usepackage{eucal,times,amsmath,amsthm,amssymb,mathrsfs,stmaryrd,color,enumerate,accents}
\usepackage{tensor}

\usepackage{textcomp}
%\RequirePackage[l2tabu,orthodox]{nag} %detect whether obsolete packages are used
%\documentclass[10pt,a4paper,reqno]{amsart} %reqno places equation numbers on the right
%\linespread{1.1}
\usepackage{mathtools,bm,eucal} % math related
%\usepackage{microtype,fixltx2e,lmodern} % latex technical issues
%\usepackage[utf8]{inputenc} % input encoding
\usepackage[T1]{fontenc} % font encoding
\usepackage{enumerate,comment,braket,xspace,csquotes} % utilities
%\usepackage[all,cmtip]{xy} % because the tikzcd options [shift left], [shift right] do not work on arXiv, we switched some diagrams to xymatrix
%\usepackage[centering,vscale=0.7,hscale=0.8]{geometry}
%\usepackage[hidelinks]{hyperref}

\numberwithin{equation}{subsection}
\newcommand{\heart}{\ensuremath\heartsuit}
%BEGIN MY COMMANDS
\newtheorem{theorem}{Theorem}[subsection]
\newtheorem{lemma}[theorem]{Lemma}
\newtheorem{corollary}[theorem]{Corollary}
\newtheorem{proposition}[theorem]{Proposition}
\newtheorem{conjecture}[theorem]{Conjecture}
\newtheorem{question}[theorem]{Question}
\newtheorem{assumption}[theorem]{Assumption}
\newtheorem{claim}[theorem]{Claim}
\newtheorem{thm-intro}{Theorem}

\newtheorem{ap-theorem}{Theorem}
\newtheorem{ap-corollary}{Corollary}
\newtheorem{ap-lemma}{Lemma}
\newtheorem{ap-proposition}{Proposition}
\newtheorem{ap-definition}{Definition}


\theoremstyle{definition}
\newtheorem{remark}[theorem]{Remark}
\newtheorem{example}[theorem]{Example}
\newtheorem{notation}[theorem]{Notation}
\newtheorem{definition}[theorem]{Definition}
\newtheorem{construction}[theorem]{Construction}
\newtheorem{warning}[theorem]{Warning}
\newtheorem{convention}[theorem]{Convention}
%\newtheorem{exercise}[theorem]{Exercise}

\renewcommand{\labelenumi}{(\roman{enumi})}
\renewcommand{\labelitemi}{-}

\newcommand{\cXysquare}[8]{
\[\cXymatrix{
#1 \ar@{#5}[r] \ar@{#6}[d] & #2 \ar@{#7}[d]\\
#3 \ar@{#8}[r] & #4
}\]
}

\DeclareMathOperator*{\indlimf}{``\colim''}
\DeclareMathOperator*{\projlimf}{``\lim''}
\DeclareMathOperator*{\holim}{\operatorname*{holim}}
\def\lim{\mathrm{lim}}
\DeclareMathOperator*{\hocolim}{\operatorname*{hocolim}}
\DeclareRobustCommand{\SkipTocEntry}[5]{}


%\newcommand{\abs}[1]{\vert#1\vert}

\ifpersonal
\newcommand{\personal}[1]{\textcolor[rgb]{0,0,1}{(Personal: #1)}}
\newcommand{\todo}[1]{\textcolor{red}{(Todo: #1)}}
\else
\newcommand*{\personal}[1]{\ignorespaces}
\newcommand*{\todo}[1]{\ignorespaces}
\fi

\newcommand{\abs}[1]{\vert#1\vert}


% Z_p-adic geometry
\newcommand{\bZp}{\bZ_p}
\newcommand{\adSt}{\mathrm{dSt}^\ad}
\newcommand{\cycbZp}{\bZp^\cyc}
\newcommand{\Ainf}{\mathrm{A}_{\mathrm{inf}}}
\newcommand{\AdR}{\mathrm{A}_{\dR}}


%groups
\newcommand{\GL}{\mathrm{GL}}
\newcommand{\GLn}{\mathrm{GL}_n}
\newcommand{\Gk}{G_K}
\newcommand{\Gal}{\mathrm{Gal}}
\newcommand{\anGLn}{\mathbf{GL}_n^\an}

%stacks
\newcommand{\fib}{\mathrm{fib}}
\newcommand{\dLocSys}{\mathrm R \mathrm{LocSys}_{\ell, n}}
\newcommand{\abdLocSys}{\mathrm R \mathrm{LocSys}_{\ell, n, \Gamma}}
\newcommand{\LocSys}{\mathrm{LocSys}_{\ell, n}}
\newcommand{\abLocSys}{\mathrm{LocSys}_{\ell, n, \Gamma}}
\newcommand{\LocSysfr}{\LocSys^{\mathrm{framed}}}
\newcommand{\abLocSysfr}{\abLocSys^{\mathrm{framed}}}
\newcommand{\cLocSys}{\mathrm{LocSys}_{\bC, n}}
\newcommand{\Conn}{\mathrm{Conn}_{\bC, n }}
\newcommand{\Hilb}{\mathrm{Hilb}}
\newcommand{\Vect}{\mathrm{Vect}}
\newcommand{\bG}{\rmB G}

%symbols
\newcommand{\FormalModels}{\mathrm{FM}}
\newcommand{\Idem}{\mathrm{Idem}}
\newcommand{\idem}{\mathrm{idem}}
\newcommand{\nil}{\mathrm{nil}}
\newcommand{\cofib}{\mathrm{cofib}}
\newcommand{\id}{\mathrm{Id}}
\newcommand{\fr}{\widehat{\rmF}_r}
\newcommand{\Higgs}{\mathrm{Higgs}}
\newcommand{\ev}{\mathrm{ev}}
\newcommand{\sm}{\mathrm{sm}}
\newcommand{\tr}{\mathrm{tr}}
%\newcommand{\dim}{\mathrm{dim}}
\newcommand{\Frob}{\mathrm{Frob}}
\newcommand{\barX}{\overline{X}}
\newcommand{\pr}{\mathrm{pr}_1}
\newcommand{\Map}{\mathrm{Map}}
\newcommand{\fpqc}{\mathrm{fpqc}}
\newcommand{\Hom}{\mathrm{Hom}}
\newcommand{\ad}{\mathrm{ad}}
\newcommand{\Ex}{\mathrm{Ex}}
\newcommand{\fCoh}{\mathfrak{Coh}^+}
\newcommand{\cyc}{\mathrm{cyc}}
\newcommand{\inf}{\mathrm{inf}}

\newcommand{\cotimes}{\widehat{\otimes}}
\newcommand{\sX}{\mathscr{X}}
\newcommand{\sY}{\mathscr{Y}}
\newcommand{\sZ}{\mathscr{Z}}

\newcommand{\sfX}{\mathsf{X}}
\newcommand{\sfY}{\mathsf{Y}}
\newcommand{\sfZ}{\mathsf{Z}}
\newcommand{\sfW}{\mathsf{W}}

\newcommand{\Ok}{k^\circ}


\DeclareMathOperator{\Spec}{Spec}
\DeclareMathOperator{\Spf}{Spf}
\DeclareMathOperator{\Sp}{Sp}
%\newcommand{\Spf}{\mathrm{Spf}}
%\newcommand{\Spec}{\mathrm{Spec}}
\newcommand{\st}{\mathrm{st}}
\newcommand{\Der}{\mathrm{Der}}
\newcommand{\trun}{\mathrm{t}}
\newcommand{\spe}{\mathrm{sp}}

\newcommand{\bfA}{\mathfrak{A}_{\Ok}}
\newcommand{\anA}{\mathbf{A}_k}
\newcommand{\anB}{\mathbf{B}_k}
\newcommand{\banA}{\mathbf{A}_{\Ok}}

\newcommand{\cEnd}{\cE \mathrm{nd}}
\newcommand{\End}{\mathrm{End}}
\newcommand{\Mat}{\mathrm{Mat}}
\newcommand{\unr}{\mathrm{unr}}
\newcommand{\Frac}{\mathrm{Frac}}


\newcommand{\tamepi}{\pi_1^{\mathrm{t}}}
\newcommand{\wildpi}{\pi_1^w}
\newcommand{\tame}{\mathrm{tame}}
\newcommand{\Mon}{\mathrm{Mon}}
\newcommand{\grp}{\mathrm{grp}}
\newcommand{\dSt}{\mathrm{dSt}}

\newcommand{\fc}{\mathrm{fc}}
\newcommand{\Sh}{\mathrm{Sh}}
\newcommand{\Ad}{\mathrm{Ad}}

\newcommand{\Def}{\mathbf{\mathrm{Def}}}
\newcommand{\art}{\mathrm{art}}
\newcommand{\cont}{\mathrm{cont}}
\newcommand{\Q}{\mathbb{Q}}
\newcommand{\Ql}{\bQ_\ell}
\newcommand{\adL}{\mathbb{L}^{\ad}}

\newcommand{\Aut}{\mathrm{Aut}}
%homotopy types


%pregeometries and local structures and algebraic categories

\newcommand{\Tdisc}{\mathcal{T}_{\mathrm{disc}}}
\newcommand{\Tet}{\mathcal{T}_{\text{\'et}}}
\newcommand{\Tetph}{\mathcal{T}_{\emph{\text{\'et}}}}
\newcommand{\Tzar}{\mathcal{T}_{\mathrm{Zar}}}
\newcommand{\Tan}{\mathcal{T}_{\an}(k)}
\newcommand{\Tad}{\mathcal{T}_{\ad}(\Ok)}
\newcommand{\cTad}{\mathcal{T}_{\ad}(\Ok)}
\newcommand{\sm}{\mathcal{C}\mathrm{Alg}^\sm}
\newcommand{\CAlg}{\mathcal{C}\mathrm{Alg}}
\newcommand{\adCAlg}{\cC \mathrm{Alg}^{\ad}_{\bZp}}
\newcommand{\CAlgad}{\cC \mathrm{Alg}^{\ad}}
\newcommand{\admCAlg}{\cC \mathrm{Alg}^{\mathrm{adm}}_{\Ok}}
\newcommand{\fCAlg}{\mathrm{f} \cC \mathrm{Alg}_{\Ok}}
\newcommand{\ftaftCAlg}{\fCAlg^{\taft}}
\newcommand{\taft}{\mathrm{taft}}
\newcommand{\AnRing}{\mathrm{AnRing}_k}
\newcommand{\wCAlg}{\cC \mathrm{Alg}^{\wedge}_{\Ok}}



\newcommand{\Str}{\mathrm{Str}}
\newcommand{\locStr}{\mathrm{Str}^{\mathrm{loc}}}
\newcommand{\adm}{\mathrm{adm}}


\newcommand{\PSh}{\mathrm{PSh}}
\newcommand{\PrL}{\mathcal{P} \mathrm{r}^{\mathrm L}}
\newcommand{\Cat}{\mathcal{C}\mathrm{at}_\infty}
\newcommand{\Coh}{\mathrm{Coh}^+}
\newcommand{\Coheart}{\mathrm{Coh}^{+, \heartsuit}}
\newcommand{\Cohb}{\mathrm{Coh}^{\mathrm b}}
\newcommand{\bCoh}{\mathrm{Coh}^\mathrm{b}}
\newcommand{\cHom}{\mathcal{H} \mathrm{om}}
\newcommand{\loc}{\mathrm{loc}}



%infty categories, spaces, modules
\newcommand{\op}{\mathrm{op}}
\newcommand{\der}{\mathrm{der}}
\newcommand{\fSpAb}{\mathrm{Sp} \left( \mathrm{Ab}( \fstrfs) \right) }
\newcommand{\Psh}{\mathrm{PShv}}
\newcommand{\Shv}{\mathrm{Shv}}
\newcommand{\ind}{\mathrm{Ind}}
\newcommand{\Ind}{\mathrm{Ind}}
\newcommand{\pro}{\mathrm{Pro}}
\newcommand{\Perf}{\mathrm{Perf}}
\newcommand{\DM}{Deligne-Mumford stack}
\newcommand{\inftopos}{$\infty$-topos\xspace}
\newcommand{\infcat}{$\infty$-category\xspace}
\newcommand{\infcats}{$\infty$-categories\xspace}
\newcommand{\sMap}{\underline{\Map}}




\newcommand{\PerfSys}{\mathrm{PerfSys}_{\ell}}
\newcommand{\rigCat}{\mathcal{C}\mathrm{at}^{\mathrm{st}, \omega, \otimes}_{\infty}}



%cohomology
\newcommand{\coho}{C^*_{\mathrm{cont}}( \K, \mathrm{Ad}(\rho))}\newcommand{\cohon}{C^*_{\mathrm{cont}}( \K, \mathrm{Ad}(\rho_n))}
\newcommand{\proet}{\text{pro\'et}}
\newcommand{\et}{\text{\'et}}
\newcommand{\emphet}{\emph{\text{\'et}}}
\newcommand{\Sym}{\mathrm{Sym}}
\newcommand{\dR}{\mathrm{dR}}

%functors
\newcommand{\tcomp}{(-)^\wedge_t}
\newcommand{\functor}{(-)}
\newcommand{\rigg}{(-)^{\mathrm{rig}}}
\newcommand{\rig}{\mathrm{rig}}
\newcommand{\alg}{\mathrm{alg}}
\newcommand{\Fun}{\mathrm{Fun}}
\newcommand{\Hub}{(-)^+}
\newcommand{\sh}{\mathrm{sh}}
\newcommand{\disc}{\functor^\mathrm{disc}}


%for categories of geometric objects
\newcommand{\Top}{\tensor[^{\mathrm{R}}]{\cT \op}{}}
\newcommand{\TopT}{\tensor[^{\mathrm{R}}]{\cT \op}{}( \Tau)}
\newcommand{\adTop}{\tensor[^{\mathrm{R}}]{\cT \op}{}( \Tad)}
\newcommand{\anTop}{\tensor[^{\mathrm{R}}]{\cT \op}{}( \Tan)}
\newcommand{\etTop}{\tensor[^{\mathrm{R}}]{\cT \op}{} (\Tet(\Ok))}
\newcommand{\etphTop}{\tensor[^{\mathrm{R}}]{\cT \op}{} (\Tetph(\Ok))}
\newcommand{\etTopk}{\tensor[^{\mathrm{R}}]{\cT \op}{}(\Tet(k))}
\newcommand{\discTop}{\tensor[^{\mathrm{R}}]{\cT \op}{}(\Tdisc(\Ok))}
\newcommand{\discTopk}{\tensor[^{\mathrm{R}}]{\cT \op}{}(\Tdisc(k))}
\newcommand{\discTopn}{\tensor[^{\mathrm{R}}]{\cT \op}{} (\Tdisc(\Ok_n))}

\newcommand{\Mod}{\mathrm{Mod}}
\newcommand{\St}{\mathrm{St}}

\newcommand{\Afd}{\mathrm{Afd}}
\newcommand{\Afdl}{\mathrm{Afd}_{\Q_\ell}}
\newcommand{\An}{\mathrm{An}}
\newcommand{\Anl}{\mathrm{An}_{\Q_\ell}}
\newcommand{\dAnl}{\mathrm{dAn}_{\Q_\ell}}
\newcommand{\dAfd}{\mathrm{dAfd}}
\newcommand{\dAfdl}{\mathrm{dAfd}_{\Q_\ell}}
\newcommand{\dAn}{\mathrm{dAn}}
\newcommand{\Aff}{\mathrm{Aff}}
\newcommand{\dAff}{\mathrm{dAff}}
\newcommand{\dSch}{\mathrm{dSch}_{\bZp}}
\newcommand{\Sch}{\mathrm{Sch}_k}
\newcommand{\dfSch}{\mathrm{dfSch}}
\newcommand{\fSch}{\mathrm{fSch}_{\Ok}}
\newcommand{\dfAff}{\mathrm{dfAff}_{\Ok}}
\newcommand{\dfDM}{\mathrm{dfDM}}
\newcommand{\fDM}{\mathrm{fDM}_{\Ok}}
\newcommand{\an}{\mathrm{an}}
%\newcommand{\Sp}{\mathrm{Sp}}
\newcommand{\Ab}{\mathrm{Ab}}
%\newcommand{\Set}{\mathrm{Set}}

%rings
\newcommand{\Zl}{\mathbb{Z}_{\ell}}


\DeclareMathOperator*{\colim}{colim}

% \mathrm
\newcommand{\rmA}{\mathrm{A}}
\newcommand{\rmB}{\mathrm{B}}
\newcommand{\rmC}{\mathrm C}
\newcommand{\rmD}{\mathrm D}
\newcommand{\rmE}{\mathrm E}
\newcommand{\rmF}{\mathrm F}
\newcommand{\rmG}{\mathrm G}
\newcommand{\rmH}{\mathrm H}
\newcommand{\rmI}{\mahtrm I}
\newcommand{\rmJ}{\mathrm J}
\newcommand{\rmL}{\mathrm{L}}
\newcommand{\rmM}{\mathrm M}
\newcommand{\rmP}{\mathrm P}
\newcommand{\rmR}{\mathrm R}
\newcommand{\rmT}{\mathrm T}

% \mathbb
\newcommand{\bA}{\mathbb A}
\newcommand{\bB}{\mathbb B}
\newcommand{\bC}{\mathbb C}
\newcommand{\bD}{\mathbb D}
\newcommand{\bE}{\mathbb E}
\newcommand{\bF}{\mathbb F}
\newcommand{\bL}{\mathbb L}
\newcommand{\bP}{\mathbb P}
\newcommand{\bN}{\mathbb N}
\newcommand{\bQ}{\mathbb Q}
\newcommand{\bR}{\mathbb{R}}
\newcommand{\bT}{\mathbb T}
\newcommand{\bZ}{\mathbb Z}

%\mathfrak
\newcommand{\fA}{\mathfrak A}
\newcommand{\fB}{\mathfrak B}
\newcommand{\fC}{\mathfrak C}
\newcommand{\fF}{\mathfrak F}
\newcommand{\fG}{\mathfrak G}
\newcommand{\ff}{\mathfrak f}
\newcommand{\fU}{\mathfrak U}
\newcommand{\fX}{\mathfrak X}
\newcommand{\fY}{\mathfrak Y}
\newcommand{\fZ}{\mathfrak Z}
\newcommand{\fg}{\mathfrak g}
\newcommand{\fm}{\mathfrak{m}}
\newcommand{\fl}{\mathfrak l}

%\mathcal
\newcommand{\cA}{\mathcal A}
\newcommand{\cB}{\mathcal B}
\newcommand{\cC}{\mathcal C}
\newcommand{\cD}{\mathcal D}
\newcommand{\cE}{\mathcal E}
\newcommand{\cF}{\mathcal{F}}
\newcommand{\cG}{\mathcal{G}}
\newcommand{\cI}{\mathcal{I}}
\newcommand{\cH}{\mathcal{H}}
\newcommand{\cK}{\mathcal{K}}
\newcommand{\cL}{\mathcal{L}}
\newcommand{\cM}{\mathcal M}
\newcommand{\cN}{\mathcal N}
\newcommand{\cO}{\mathcal{O}}
\newcommand{\cP}{\mathcal{P}}
\newcommand{\cQ}{\mathcal{Q}}
\newcommand{\cV}{\mathcal V}
\newcommand{\cT}{\mathcal{T}}
\newcommand{\cX}{\mathcal X}
\newcommand{\cY}{\mathcal Y}
\newcommand{\cZ}{\mathcal Z}
\newcommand{\cS}{\mathcal S}

%\widehat

\newcommand{\hB}{\widehat{B}}
\newcommand{\hAA}{\widehat{A'}}
\newcommand{\hAp}{\widehat{A}_X[p^{-1}]}
\newcommand{\hbZ}{\widehat{\bZ}}
%\newcommand{\abs}[1]{\vert#1\vert}
\newcommand{\overK}{\overline{K}}

%\widetilde

\newcommand{\tP}{\widetilde{P}}
\newcommand{\tQ}{\widetilde{Q}}
\newcommand{\tR}{\widetilde{R}}
\newcommand{\tS}{\widetilde{S}}
\newcommand{\tT}{\widetilde{T}}
\newcommand{\tV}{\widetilde{V}}
\newcommand{\tW}{\widetilde{W}}
\newcommand{\tX}{\widetilde{X}}
\newcommand{\tY}{\widetilde{Y}}
\newcommand{\tZ}{\widetilde{Z}}

%\Cech
\newcommand{\vC}{\textrm{\v{C}}}




\author{Jorge ANT\'ONIO}
\address{Jorge Ant\'onio,  Institut de Math\'ematiques de Toulouse, 118 Rue de Narbonne  31400 Toulouse}
\email{\texttt{jorge\_tiago.ferrera\_antonio@math.univ-toulouse.fr}}

\author{Joost NUITEN}
\address{Joost Nuiten, IMAG Universit\'e de Montpellier, Place Eug\`{e}ne Bataillon
34090 Montpellier}


\date{}

\title{A Van Est theorem in mixed characteristic} 



\begin{document}


\maketitle

\personal{PERSONAL COMMENTS ARE SHOWN!!!}

\tableofcontents

\section{Geometric context}
\personal{Does one really needs to work locally for the fpqc topology or can we refine the topology further by dropping the quasi-compactness assumption?}

\subsection{$\bZp$-adic geometric stacks}

Let $\adCAlg$ denote the \infcat of simplicial $\bZp$-adic algebras. 

\begin{definition}
    Let $f \colon A \to B$ be a morphism in the \infcat $\adCAlg$. The morphism $f$ is said to be \emph{\'etale} (resp., \emph{smooth}) if it is \emph{almost topologically of finite presentation} and the relative cotangent complex
        \[
                \adL_{B/A} \in \Mod_B.   
        \]
    vanishes (resp. it is equivalent to a free $B$-module of finite rank concentrated in degree $0$).
\end{definition}

\begin{definition}
    Let $\rmP_\sm$ denote the class of smooth morphisms in the \infcat $\adCAlg$. The triplet $(\adCAlg, \tau_\et, \rmP_\sm)$ forms a \emph{geometric context}, which we refer to as the \emph{derived $\bZp$-adic geometric context}.
\end{definition}

\begin{definition}
    The \infcat of \emph{derived $\bZp$-adic geometric stacks} is defined as the \infcat of geometric stacks
        \[
             \adSt \coloneqq \dSt( \adCAlg, \tau_\et, \rmP),
        \]
    with respect to the derived $\bZp$-adic geometric context.
\end{definition}

\begin{remark}
    Let $
            L_p^\wedge \colon \CAlg_{\bZp} \to \adCAlg
        $
    denote the \emph{$p$-completion functor}, introduced in \cite[\S 8]{lurie2016spectral}. Given $A \in \adCAlg$, we define $A_n \in \adCAlg$ as the pushout
        \[
        \begin{tikzcd}
            A[t] \ar{r}{t \mapsto p^n} \ar{d}{t \mapsto 0} & A \ar{d} \\
            A \ar{r} & A_n
        \end{tikzcd}
        \]    
    computed in \infcat $\CAlg_{\bZp}$. Notice that $A_n$ is naturally an object of the \infcat $\adCAlg$ via the canonical inclusion $\adCAlg \subseteq \CAlg_{\bZp}$.
    Thanks to \cite[Lemma 8.1.2.3]{lurie2016spectral}, one has a natural equivalence
        \[
            (A)^\wedge_p \simeq \lim_{\geq 1} A_n,  
        \]
    in the \infcat $\adCAlg$, where $(A)^\wedge_p$ denotes the $p$-completion of $A$.
    Given a functor $X \colon \CAlg_{\bZp} \to \cS$, we define its \emph{$p$-completion} as the functor
        \[
                X^\wedge_p \colon \adCAlg \mapsto \cS  
        \]
    given by the formula
        \begin{equation} \label{p-comp_stacks]}
                A \in \adCAlg \mapsto \lim_{n \geq 1} X(A_n) \in \cS.  
        \end{equation}
    From the above formula \eqref{p-comp_stacks}, it is clear that if $X$ satisfies \'etale hyper-descent then so it does $X^\wedge_p$ satisfies descent with respect to $\tau_\et$-hypercoverings in $(\adCAlg, \tau_\et)$.
    Suppose now that $X \in \dSt( \CAlg_{\bZp}, \tau_\et, \rmP_\sm)$ denotes an (algebraic) derived $\bZp$-geometric stack. Then the $p$-completion of the corresponding functor of points,
    which we shall simply denote by $X^\wedge_p$, is naturally a derived $\bZp$-adic geometric stack, i.e., $X^\wedge_p \in \adSt$.
    To see this, let
        \[
                \pi \colon P \to X  ,
        \]
    be a smooth covering of $X$ where $P \in \dSch$ is a derived $\bZp$-scheme. Then
        \[
                (
                \pi
                )^\wedge_p \colon P^\wedge_p \to X^\wedge_p  
        \]
    is still a smooth covering of $X^\wedge_p$ by the derived $\bZp$-adic scheme, $P^\wedge_p$.

    More generally, given a compatible \emph{ind-system}, $\{X_n \}_{n \geq 1}$, where for each $n \geq 1$, $X_n$ denotes a derived $\bZ/p^n$-geometric stack. Then the induced functor
        \[
            \colim_{n \geq 1} X_n \colon \adCAlg \to \cS  
        \]
    is naturally a derived $\bZp$-adic geometric stack.
\end{remark}

\begin{example}
    Let $X \in \dSch$ denote a derived $\bZp$-scheme, then its $p$-completion, $X^\wedge_p$ is naturally an object in the \infcat $\adSt$. Of interest to us will be the derived moduli stack parametrizing perfect complexes, $\Perf$ and classifying spaces of formal groups or $p$-divisible groups.
    Given any formal reductive group $G$, $\mathrm{Bun}_G$ also lives naturally in the \infcat $\adSt$.
\end{example}

\begin{remark}
    Let $X \in \St(\CAlg_{\bZp}, \tau_\et, \rmP_\sm)$. Thanks to Artin-Lurie representability theorem, \cite[?]{lurie2012dag}, the functor of points associated to $X$ is nilcomplete, infinitesimally cartesian
    and it admits a global algebraic cotangent complex, which is an almost perfect complext on $X$. From the definitions, it is clear that $X^\wedge_p$ also possedes these properties, namely it is nilcomplete, infinitesimally cartesian and it admits a global almost perfect $\bZp$-adic cotangent complex. Moreover,
    one has a natural equivalence
        \[
                \adL_{X^\wedge_p} \simeq (\bL_X  )^\wedge_p \in \Mod_X.
        \]
    \todo{Actually, state this in a precise way.}
\end{remark}

\personal{One would like to have a Representability theorem in the context of derived $\bZp$-adic geometry. Otherwise, it will be difficult to state precisely that geometric stacks like $\Perf$ satisfy geometricity. Another way around this problem, might be by identifying such moduli
with completion along the ideal $(p) \subseteq \bZp$.}
\personal{Notice that Artin-Lurie representability holds true for Noetherian $\bE_\infty$-rings such that $\pi_0(R)$ is a Grothendieck ring. In particular, $\bZp$ is an example of such so we do have Artin-Lurie Representability theorem for \emph{algebraic} derived geometric $\bZp$-stacks.}


\personal{Notice that we would like that our statements are true more generally. In particular, we would like to be able to treat the case of the moduli of $p$-divisible groups.
However, this object is not geometric in our sense, since it does not admit a smooth covering. The only obstruction, it seems at this point, is to prove the existence of the local section, since for this we admit that we have
a smooth covering of the form $P \to Y$.}

\begin{definition}
    Let $X \colon \adCAlg \to \cS$. We say that $X$ satisfies \emph{faithfully flat-descent} if for every faithfully flat morphism
        \[
           f \colon  A \to B  ,
        \]
    in the \infcat $\adCAlg$, one has a canonical equivalence
        \[
            X(\vC(B/A)) \to  X(A) 
        \]
    in the \infcat $\cS$, where $\vC(B/A)$ denotes the \emph{$p$-complete $\vC$ech nerve} associated to the morphism $f \colon A \to B$.
    \todo{Define the $p$-complete Cech nerve.}
\end{definition}

\begin{lemma}
    Let $X \in \Shv(\CAlg_{\bZp}, \tau_\et)$ denote an \'etale sheaf with respect to the site $(\adCAlg, \tau_\et)$. Assume further that $X$ satisfies \emph{fpqc}-descent. Then the $p$-completion
        \[
                X^\wedge_p \colon \adCAlg \to \cS  
        \]
    also satisfies $\fpqc$-descent.
\end{lemma}

\begin{proof}
    The result is an immediate consequence of the fact that limits preserve limits. \todo{expand this proof.}
\end{proof}

\subsection{Integral perfectoid algebras}

Let $R \in \adCAlg$ denote a $\pi$-adically complete and separated, for some element $\pi \in R$, (discrete) $\bZp$-algebra. Denote by
    \[
        \varphi \colon R/ pR \to R/p  R
    \]
the \emph{absolute Frobenius} of $R/pR$. We define the \emph{tilt} of $S$, denoted $S^\flat$, as the inverse limits
    \[
        S^\flat \coloneqq \lim_{\varphi} R/ pR, 
    \]
which is a \emph{perfect $\bF_p$-algebra.}

\begin{definition}
    We define \emph{Fontaine's ring} associated to $R$ as
        \[
                \Ainf(R) \coloneqq W(R^\flat).
        \]
    \personal{Maybe one can define $\Ainf(R)$ already as the pro-completion over the kernel of the projection map $\Ainf(R) \to R$.}
\end{definition}


\begin{definition}
Let $R \in \adCAlg$. We say that $R$ is an \emph{integral perfectoid $\bZp$-algebra} if if satisfies the following conditions:
    \begin{enumerate}
            \item There exists an element $\pi \in R$ such that $R$ is $\pi$-adically complete and such that $\pi^p \vert p$;
            \item The Frobenius morphism    
                        \[
                                \varphi \colon R/p R \to R/ pR
                        \]  
                  is surjective;
            \item The kernel of the canonical morphism $\theta \colon \Ainf(R) \to R$ is generated by a single element.
    \end{enumerate}
\end{definition}

\begin{example} The following is a transcription of \cite[Example 3.6]{bhatt2018integral}.
    If $R $ is such that $pR = 0$, then $R$ is necessarily perfect. Let $\cycbZp \coloneqq (\bZp[\zeta_{p^\infty}])^\wedge_p$ denote the $p$-adic completion of the ring of integers of the cyclotimic extension $\bQ_p(\zeta_{p^\infty}) / \bQ_p$. Then $\bZp^\cyc$ is a perfectoid $\bZp$-algebra.
    Replacing $\bQ_p(\zeta_{p^\infty})$ by any other algebraic extension of it and taking the ring the $p$-adic completion of its ring of integers provides also an integral perfectoid $\bZp$-algebra.
    The $p$-adic complete ring $\bZp^\cyc \langle T^{1/p^\infty} \rangle$ is itself a perfectoid $\bZp$-algebra. So it is the perfectoid torus, $\bZp^\cyc \langle T^{\pm 1/p^\infty} \rangle$.
\end{example}

\begin{lemma}
    Let $R \in \adCAlg$ be an integral perfectoid $\bZp$-algebra which is $p$-torsion free. Then $R$ is (derived) $p$-complete.
\end{lemma}

\begin{proof}
    Let $R \in \adCAlg$ be an integral perfectoid $\bZp$-algebra which we assume further to be $p$-torsion free. We wish to show that $R$ is $p$-complete. Thanks to \cite[Lemma 8.1.2.3]{lurie2016spectral}, it suffices to show that the canonical map
        \[
            R \to \lim_{n \geq 1} R_n
        \]  
    is an equivalence in the \infcat $\adCAlg$. In our situation, we can identify, for each $n \geq 1$, 
        \[
            R_n \simeq R \otimes_{\bZp} \bZ/ p^n\bZ .
        \]
    Therefore, for each integer $i \geq 0$, $\pi_i(R_n) = 0$, except in the case where $i=0, \ 1$, in which we have $\pi_1(R_n) = R[p^n] =0$ and $\pi_0(R_n) = R/p^n$. Passing to inverse limits, the Milnor short exact sequence implies that $\pi_i(R^\wedge_p) =0$ for $i>0$. For $i=0$, we have a Milnor short exact sequence of the form
        \begin{equation} \label{p-tors_perf}
            0 \to \lim^1_{n \geq 1} \pi_0(R_n) \to \pi_0(R^\wedge_p) \to \lim_{n \geq 1} \pi_0(R_n)= R \to 0.
        \end{equation}
    Since the the transition maps in the pro-system $\{ R/p^n \}_{n \geq 1}$ are surjective, it follows that the left hand side in \eqref{p-tors_perf} vanishes. Therefore, one obtains an equivalence $R^\wedge_p \simeq R$, as desired.
\end{proof}

The following result is of fundamental importance for our purposes in the current text:

\begin{proposition} \label{prop_non_obs_perf}
    Let $R \to S$ be a morphism between integral perfectoid $\bZp$-algebras in the \infcat $\adCAlg$. Then $\adL_{S/R} \simeq 0$.
\end{proposition}

\begin{proof}
    This is the content of \cite[Proposition 3.14]{bhatt2018integral}.
\end{proof}

\begin{remark}
    Let $f \colon R \to S$ be a morphism between integral perfectoid $\bZp$-algebras in the \infcat $\adCAlg$. Thanks to \cref{prop_non_obs_perf} it follows, in particular, that
        \[
                \adL_{S/ R} \otimes_{R} R_1 \simeq 0  .
        \]
    As a consequence, the morphism $f \colon R \to S$ is \emph{formally \'etale}. As a consequence, the morphism induced by $f $, at the level of tiltings,
        \[
               \overline{f}^\flat \colon (R/p)^\flat   \to (S/ p)^\flat
        \]
    lifts, by taking the Witt-vectors construction, to a deformation $f_\mathrm{inf} \colon \Ainf(R) \to \Ainf(S)$ which is the unique (up to isomorphism) such deformation of the morphism $f$. Moreover, since 
        \[
            S \simeq \Ainf(S) \otimes_{\Ainf(R)} R
        \]
    in the \infcat $\adCAlg$ it follows that $f_{\inf}$ is the universal
    deformation of $f$.
\end{remark}

\begin{lemma}
    Let $R \in \adCAlg$ denote an integral perfectoid $\bZp$-algebra. Then $\Ainf(R) \in \adCAlg$ is $p$-torsion free.
\end{lemma}

\begin{proof}
    By construction, we have that $\Ainf(R) = W((R/pR)^\flat)$. The $\bF_p$-algebra $(R/p)^\flat$ is a perfect $\bF_p$-algebra. For this reason, $W((R/pR)^\flat)$ is $p$-torsion free.
    \todo{This result is certainly true, however a reference is still needed here.}
\end{proof}

The following two results will establish the precise relation between $\Ainf(R) $ and $R$, via obstruction theory, for perfectoid $R$:

\begin{lemma}
    The morphism $\theta \colon \Ainf(R) \to R$ is a pro-thickening morphism.
\end{lemma}

\begin{proof}
    This is a result due Fontaine.
\end{proof}

\begin{proposition} \label{Ainf_de_Rham}
    Let $R \in \adCAlg$ be a $p$-torsion free integral perfectoid $\bZp$-algebra. Then for every $p$-complete $A \in \adCAlg$ we have a natural equivalence of mapping spaces
        \[
               \theta \colon \Map_{\adCAlg}(\Ainf(R), A) \simeq   \Map_{\adCAlg}(R, \pi_0(A)_\mathrm{red}) .
        \]
\end{proposition}

\begin{proof}
    Let $f \colon R \to \pi_0(A)_\mathrm{red}$ denote a continuous $p$-adic morphism. Given any continuous morphism $f \colon R \to \pi_0(A)_\mathrm{red}$ we obtain, by base change along the morphism $\bZp \to \bF_p$, a well defined morphism
        \[
            \overline{f}_{\mathrm{red}, 1} \colon (R/p)^\flat \to \pi_0(A)_{\mathrm{red}, 1} . 
        \]
    Since $(R/p)^\flat $ is a perfect $\bF_p$-algebra, it follows that $\overline{f}_1$ lifts canonically to a uniquely (up to isomorphism) defined morphism of $\bF_p$-algebras
        \[
            \overline{f}_1 \colon (R/p)^\flat \to \pi_0(A)_1.  
        \]
    We shall prove, by induction on the Postnikov tower for $A \in \adCAlg$, that $overline{f}_1$ lifts uniquely (up to a contractible space of indeterminacy) to a well defined morphism
        \[
            \overline{f}_1 \colon (R/p)^\flat \to A_1.
        \]
    The case $n=0$ has already been dealt with. Suppose now that for $n \geq 0$ we have constructed a unique, up to contractible indeterminacy, morphism
        \[
            \overline{f}(n) \colon (R/p)^\flat \to \tau_{\leq n} A_1  
        \]
    in the \infcat $\adCAlg$. Indeed, consider the natural derivation, at level $n$ for $A$,
        \[
                d_n \colon \tau_{\leq n} A_1 \to \tau_{\leq n}A_1 \oplus \pi_{n+1}(A_1)[n+2].
        \]
    Pre-compositoin with $\overline{f}_1(n) \colon (R/p)^\flat \to \tau_{\leq n}A_1$ induces a morphism 
        \[
                (R/p)^\flat \to \tau_{\leq n}A_1 \to \tau_{\leq n}A   \oplus \pi_{n+1}(A_1)[n+2],
        \]
    over $\tau_{\leq n}A$. Moreover, we have canonical identification of mapping spaces
        \[
                \Map_{\big(\cC \mathrm{Alg}^\ad_{(R/p)^\flat}\big)_{/ \tau_{\leq n}(A_1) }} \big((R/p)^\flat, \tau_{\leq n}A_1 \oplus \pi_{n+1}(A_1)[n+2] \big) \simeq
                \Map_{\Mod_{(R/p)^\flat}} \bigg(\adL_{(R/p)^\flat}, \pi_{n+1}(A_1) [n+2] \bigg)
        \]
    As $(R/p)^\flat$ is a perfect $\bF_p$-algebra, it follows that $\adL_{(R/p)^\flat} \simeq 0 $. Therefore, the morphism $\overline{f}_1(n)$ lifts uniquely into a morphism a well defined morphism
        \[
                \overline{f}_1(n+1) \colon (R/p)^\flat \to \tau_{\leq n+1}A_1,  
        \]
    in the \infcat $\adCAlg$ and the inductive step is proved. Passing to inverse limits we obtain a well defined morphism
        \[
                \overline{f}_1 \colon (R/p)^\flat \to A_1
        \]
    in the \infcat $\adCAlg$.
    Moreover, as $(R/p)^\flat$ is a perfect $\bF_p$-algebra, it follows that $\Ainf(R)/p^n \simeq W_n(R/p^\flat)$ in the \infcat $\adCAlg$. By the universal property of $p$-typical Witt vectors one obtains, for each integer $n \geq 1$, uniquely defined (up to contractible indeterminacy) morphisms
        \[
            \overline{f}_n \colon \Ainf(R)/p^n \to A_n.
        \]  
    \personal{Notice that the universal property of Witt vectors is derived in nature. For example, given two objects $A, \ B \in \adCAlg$ we can find resolutions by polynomial algebras of these, seen as $\bZp$-simplicial algebras.
    We obtain thus a morphism of simplicial polynomial algebras of the form $ \varphi \colon P_\bullet \to Q_\bullet$. If $Q_\bullet$ is defined over some $\bZ_p/p^n$ then applying Witt vectors component-wise to $P_\bullet$ provides a uniquely defined (up to contractible homotopy) $W_n(P_\bullet) \to Q_\bullet$ and thus $W_n(A) \to B$ in $\adCAlg$.}
    Passing to the limit one obtains a unique lift (up to contractible indeterminacy) $\overline{f} \colon \Ainf(R) \to A$. 
    \personal{One still does not have showed the contractibility of the fiber of the morphism $\theta$, but our reasoning applies to show contractibility of higher coherences, since everything is done via universal properties.} \todo{Write this more precisely}

\end{proof}


\begin{corollary}
    Let $R \in \adCAlg$ be an integral perfectoid $\bZp$-algebra. Then $\Spf (\Ainf(R)) \simeq \big( \Spf(R) \big)_\dR$ in the \infcat $\Shv_\fpqc(\adCAlg)$.
\end{corollary}

\begin{proof}
    This is a direct consequence of \cref{Ainf_de_Rham} and the construction of the de Rham stack associated to a formal spectrum. \todo{Write this a bit better.}
\end{proof}

The following result shows that every object $A \in \adCAlg$ which is regular and Noetherian admits a faithfully flat map to a perfectoid ring:

\begin{theorem}
    Let $A \in \adCAlg$ be a discrete $\bZp$-adic algebra. Then $A$ is Noetherian and regular if and only if exists a faithfully flat morphism
        \[
                A \to A_\infty  ,
        \]
    in the \infcat $\adCAlg$, where $A_\infty \in \adCAlg$ is an integral perfectoid $\bZp$-algebra.
\end{theorem}

\begin{proof}
    This is precisely the content of the main result \cite[Theorem 4.7]{bhatt2019regular}.
\end{proof}

\begin{remark}
    Let $\bT^n \coloneqq \Spf (\bZp \langle T_1^{\pm 1}, \dots , T_n^{\pm 1} \rangle)$ denote the formal torus of dimension $n$. We define
        \[
                \bT^n(p^\infty) \coloneqq \Spf ( \cycbZp \langle T_1^{\pm 1/p^\infty}, \dots, T_n^{\pm 1/p^\infty})  .
        \]
    We have a canonical morphism
        \[
            \pi_n \colon \bT^n(p^\infty) \to \bT^n  
        \]
    which is an fpqc-covering and exhibits $\bT^n(p^\infty)$ as a $\Gamma \coloneqq \bZp(1)^n$-torsor over $\bT^n$. \todo{justify this with a refenrece. Does $\Gamma$ really have the correct dimension?}
    In this case, we construct an explicit faithfully flat covering of $\bT^n$. Moreover, suppose that we have an \'etale morphism \todo{Does \'etale really suffices or do we really finite \'etale? Check this.}
        \[
                \Spf R \to \bT^n,
        \]
    where $ R \in \adCAlg$. Then we have a perfectoid covering
        \[
                \Spf (R_\infty) \to \Spf(R)  
        \]
    where $R_\infty \coloneqq R \otimes_{\bZp \langle T_1^{\pm 1}, \dots , T_n^{\pm 1} \rangle} \cycbZp \langle T_1^{\pm 1/p^\infty}, \dots, T_n^{\pm 1/p^\infty} \rangle$.
    Then $R_\infty$ is an integral perfectoid $\bZp$-algebra,\todo{Check this!} which provides us with a faithfully flat morphism
        \[
                R \to R_\infty    
        \] 
    in the \infcat $\adCAlg$. \todo{Check that $\Spf R_\infty$ is quasi-compact.}
\end{remark}

\subsection{Smooth morphisms between derived $\bZp$-adic geometric stacks}

In this \S we wish to prove the following:

\begin{theorem}
    Let $f \colon X \to Y$ denote a smooth morphism in the \infcat $\adSt$. Then, locally on both $X$ and $Y$, $f$ can be factored as a composite
        \[
        \begin{tikzcd}
                X \ar{r}{g} & Y \times \fA^n_{\bZp} \ar{r}{\mathrm{pr}_1} & Y
        \end{tikzcd}
        \]
    where $g$ is an \'etale map of derived $\bZp$-adic geometric stacks and $\pr$ denotes the canonical projection.
\end{theorem}
\subsection{Assumptions}

First of all we will define the context in which we are going to work through.  Let us denote $\adCAlg$ the \infcat of adic $\bZp$-simplicial algebras, introduced in \cite{antonio2018p}. We consider the \infcat $\adSt(\adCAlg, \tau_{\et}, \rmP_{\sm})$ of \emph{derived $\bZp$-adic geometric stacks}.
Suppose we are given a diagram of the form
    \[
    \begin{tikzcd}
        M \ar{r}{p} \ar{d}{q} & X \\
        Y 
    \end{tikzcd}
    \]
where $M \in \dfSch$ is a smooth $\bZp$-adic scheme and $X, \ Y \in \adSt(\adCAlg, \tau_{\et}, \rmP_{\sm})$. Throughout the text, $p$ denotes a smooth surjective morphism and $q$ a smooth morphism of derived $\bZp$-adic geometric stacks. We assume further that $Y$ satisfies \emph{fpqc descent}.
\personal{The need to assume that $Y$ satisfies fpqc descent boils down to the fact that the \'etale topology is not sufficient for many of our purposes. Namely, in order to compute the correct mapping spaces of derived $\bZp$-adic geometric stacks one needs to pass to perfectoid coverings, in a similar vein as in BMS1. 
More precisely, given a Noetherian regular algebra $\bZp$-algebra $A$, by a theorem of Bhatt, Iyegar and Ma one knows that there exists a faithfully flat morphism
$A \to A_\infty$, where $A_\infty$ is integral perfectoid. Furthermore, by a celebrated result of Abhyankar one can suppose further that $A_\infty$ admits all $p$-th power roots of unity, thus living over the perfectoid covering $\cycbZp$ of $\bZp$. However, it seems that, in general, the morphism $A \to A_\infty$ is not \'etale or even weak \'etale in the sense of \cite{bhatt2013pro}.
Therefore, in order to be able to reduce our local computations to computations involving the perfectoid nature of suitable algebras, one needs to assume $Y$ to satisfy fpqc descent.
This is indeed the case when $Y = \Perf, \ \mathrm{QCoh}^\heartsuit, \ \mathrm{QCoh}, \mathrm{Bun}_G$, where $G$ denotes a reductive group, $X$ a formal scheme, $\mathrm{B} G$, for $G$ a formal group scheme or a $p$-divisible group, $\mathcal{M}_{\mathrm{FM}}$ the moduli space of $p$-divisible groups, etc.}

Our goal is to prove a mixed characteristic analogue of a theorem of Van Est, in the context of (derived) differential geometry, proved in great generality by J. Nuiten using the Koszul duality for Lie algebroids, see \cite{nuiten2017koszul}. 
\todo{Recall the statement}
\todo{Check that the geometricity of the stack $Y$ is really need or we can relax a bit the assumptions on $Y$, in order for the result to be compatible with $\cM_{\FormalModels}$, for example.}
\section{The existence of a local section}

We want to prove a connectivity statement for the canonical morphism of mapping spaces
    \[  
        \Map_{M/ }\big( X, Y \big) \to \Map_{\adSt} \big( X^\wedge_M , Y^\wedge_M \big) .
    \]

Suppose then that we are given a morphism of sheaves
    \[
        X^\wedge_M \to Y  .
    \]
Assume further that we have a smooth covering $\pi \colon P \to Y $, in the \infcat $\adSt(\adCAlg, \tau_{\text{\'et}}, \rmP_{\sm})$ and form the pullback square
    \[
    \begin{tikzcd}
            P' \ar{r} \ar{d} & P \ar{d} \\
            X^\wedge_M \ar{r}  & Y.
    \end{tikzcd}
    \]
In particular, the natural morphism $P' \to X^\wedge_M$ is a smooth covering. Moreover, we have a natural inclusion morphism   
    \[
        M \to X^\wedge_M.
    \]
Our goal is to lift the morphism $X^\wedge_M \to Y$ to a morphism $M \to P$ which induces a well defined morphism at the level of \v{C}ech nerves 
    \[
        \vC(M_\bullet / X) \to \vC(P_\bullet / Y). 
    \]
In order to do so, we will construct a section of the smooth morphism 
    \[
        P' \to X^\wedge_M,  
    \]
which induces, via composition, a morphism $M \to P$. The construction of the section is a local construction and it is precisely in this situation that we will use the setting of integral perfectoid $\bZp$-algebras.

\begin{lemma}
If such a section to $X^\wedge_M \to P'$ exists then it induces such a morphism at the level of \v{C}ech nerves
    \[
        \vC(M_\bullet / X) \to \vC(P_\bullet / Y)
    \]
\end{lemma}

\subsection{First reduction step} Let $r \colon U \to X$ denote a morphism in $\adSt( \adCAlg, \tau_{\'et}, \rmP_\sm)$ such that
    \[
        U \coloneqq \Spf A  
    \]
is a derived affined $\bZp$-adic scheme. 

\personal{Do we have to impose assumptions on the morphism $U \to X$, such as being Zariski open, \'etale?}

\begin{proposition} \label{stat_1} Up to shrienking $U$ we can suppose that the pullback of the smooth covering map $p \colon M \to X$ along $r$ can be realized as a composition of the form
    \[
    \begin{tikzcd}
        U \times \Spf R \ar{r}{g} & U \times \bT^n \ar{r}{\pr} & U ,
    \end{tikzcd}
    \]
where $g$ is an \'etale morphism. Moreover, $g$ itself factors as a composite
    \[
    \begin{tikzcd}
            U \times \Spf R \ar{r}{g'} & U \times W \ar{r}{s} & U \times \bT^n,
    \end{tikzcd}  
    \]
where $g'$ corresponds to a rational domain inclusion and $s$ is a finite \'etale map.
\end{proposition}

\begin{remark}
    The second part of the above statement might not be necessary for our purposes. However I am not sure yet.
\end{remark}

\begin{notation}
    We will denote by $\cycbZp$ the integral perfectoid $\bZp$-algebra obtained from $\bZp$ by adding all $p$-th roots of unity to $\bZp$ and performing a $p$-adic completion.
\end{notation}
Assuming \cref{stat_1} holds, one can further proceed to show that:


\begin{lemma}
    There exists a commutative diagram, \personal{possibly cartesian and fpqc coverings?}, of the form
    \begin{equation}
    \begin{tikzcd} \label{diag_chain_pulls}
            U^\cyc \times_{\Spf \cycbZp} \Spf R_\infty \ar{r} \ar{d} & U \times \Spf R \ar{d} \\  
            U^\cyc \times_{\Spf \cycbZp} \bT^n(p^\infty) \ar{r} \ar{rd} & U \times \bT^n \ar{d} \\
             & U
    \end{tikzcd}
    \end{equation}
    in the \infcat $\Shv_{\fpqc}(\adCAlg)$. Here 
        \[
            U^\cyc \coloneqq U \times \Spf \cycbZp
        \]
    and 
         \[
            \bT(p^\infty) \to \bT 
        \]
        denotes the perfectoid cover of the torus $\bT$, given by adding $p$-th roots of unity to $\bZp$ and $p$-th roots of the (free invertible) variables on $\bT$.
\end{lemma}

\begin{lemma}
    The \emph{fpqc}-covering $\bT(p^\infty) \to \bT$ is a $\Gamma \coloneqq \prod_n \bZp(1)$-torsor, for the \emph{fpqc}-topology, which is compatible with the \emph{fpqc}-$\bZp(1)$-torsor
        \[
                \Spf \cycbZp \to \Spf \bZp.  
        \]
\end{lemma}


\begin{proposition}
    The pullback square of derived $\bZp$-adic geometric stacks
        \[
        \begin{tikzcd} 
            U \times \Spf R \ar{r} \ar{d} & M \ar{d}{p} \\
            U \ar{r} & X
        \end{tikzcd}
        \]
    induces a canonical morphism $U \times ( \Spf R)_{\dR} \to X^\wedge_M$. Moreover, if the morphism $U \to X$ is surjective or a covering (namely, an effective epimorphism of sheaves) then
    so it is the canonical morphism
        \[
                U \times (\Spf R)_{\dR} \to X^\wedge_M.  
        \]
\end{proposition}

\subsection{Further reductions} Our initial situation can be thus translated as follows: consider the chain of pullback diagrams
    \[
    \begin{tikzcd}
        \tP'' \ar{r} \ar{d} & \tP' \ar{r} \ar{d} & \tP \ar{r} \ar{d} & P \ar{d} \\
        U \times \Spf R \ar{r} \ar{d} & M \ar{r} \ar{d} & X^\wedge_M \ar{r} & Y \\
        U \ar{r} & X & &
    \end{tikzcd}.
    \]
The induced morphism $U \times \Spf R \to X^\wedge_M$ factors as
    \begin{equation} \label{diag1}
            U \times \Spf R \to U \times (\Spf R)_{\dR} \to X^\wedge_M . 
    \end{equation}
Consider the following \emph{pullback} diagram in the \infcat, $\Shv_{\fpqc}(\adCAlg)$, \personal{is it really a pullback diagram?},
    \[
    \begin{tikzcd}
            \Spf R_\infty \ar{r} \ar{d}{\theta} & \Spf R \ar{d} \\ 
            \Spf \AdR (R_\infty) \ar{r} & (\Spf R)_\dR,
    \end{tikzcd}
    \]
where  $\AdR(R_\infty)$ denotes the completion with respect to the kernel (structural) ring homomorphism 
    \[
        \theta \colon \Ainf (R_\infty) \to R_\infty, 
    \]
which, by the perfectoid nature of $R_\infty$, is generated by a single elemet $\varepsilon \in \Ainf(R_\infty$.

\begin{lemma}
        Let $R_\infty \in \adCAlg$ denote a perfectoid $\bZp$-algebra and consider the (structural) morphism
            \[
                \theta \colon \Ainf(R_\infty) \to R_\infty  .
            \]
        Then $\theta$ exhibits $\Spf \AdR(R_\infty)$ as the \emph{de Rham stack} associated to $\Spf R_\infty$, $(\Spf R_\infty)_\dR$. Furthermore, if we are given a faithfully flat morphism $R \to R_\infty$, where $R$ is a regular Noetherian $\bZp$-adic algebra, then
        $\Spf \AdR(R_\infty) \to (\Spf R)_{\dR}$ is a fpqc morphism.
\end{lemma}

Restricting further along the faithfully flat morphisms $\Spf R_\infty \to \Spf R$ and $\Spf \AdR(R_\infty) \to (\Spf R)_{\dR}$ one obtains thus a pullback square of the form
    \begin{equation} \label{diag_local}
    \begin{tikzcd}
        P_\infty \ar{r} \ar{d} & \tP_\infty \ar{d} \\
        U^\cyc \times_{\Spf \cycbZp} \Spf R_\infty \ar{r} & U^\cyc \times_{\Spf \cycbZp} \Spf \AdR(R_\infty)
    \end{tikzcd}    
    \end{equation}
in the \infcat $\Shv_{\fpqc}(\adCAlg)$.

\begin{remark}
Notice that, up to shrienking $U$ mapping into $X$, one can assume that the smooth map $\tP'' \to U \times \Spf R$, displayed in diagram \eqref{diag_chain_pulls}, is actually of the form
    \[
            U \times \Spf R \times \Spf S \to U \times \Spf R ,  
    \]
where $\Spf S $ admits a suitable \'etale morphism to a torus $\Spf S \to \bT^m$. 
\end{remark}

For this reason, building upon what we have discussed so far, one can actually suppose that the diagram \eqref{diag_local} is of the form
    \[
    \begin{tikzcd}
        U^\cyc \times_{\Spf \cycbZp} \Spf S_\infty \times_{\Spf \bZp^\cyc} \Spf R_\infty \ar{r} \ar{d} & \tP \ar{d} \\
        U^\cyc \times_{\Spf \cycbZp} \Spf R_\infty \ar{r} & U^\cyc \times_{\Spf \bZp^\cyc} \Spf \AdR(R_\infty),
    \end{tikzcd}
    \]
where $S_\infty$ is a perfectoid $\bZp$-algebra which admits a faithfully flat morphism of the form $S \to S_\infty$. Moreover, as before, we can assume that $S_\infty$ admits all $p$-th roots of unity and $p$-th roots of the (invertible free) variable overs $\bT^m$ (\todo{make this assertion more precise}). In particular, we deduce that $S_\infty$ is naturally a $\cycbZp$-algebra.

\subsection{Existence and descent properties of local sections} In order to construct the desired section one would be reduced to show the following:

\begin{notation}
    We denote by $\cotimes$ the $p$-complete tensor product in the \infcat $\adCAlg$.
\end{notation} 

\begin{proposition}
    The \emph{adic cotangent complex} 
        \[
                \adL_{S_\infty \cotimes_p R_\infty / R_\infty} \simeq 0.  
        \]
    In particular, there are no obstructions to find a deformation of the morphism
        \[
                U^\cyc \times_{\Spf \cycbZp} \Spf S_\infty \times_{\Spf \cycbZp} \Spf R_\infty \to U^\cyc \times_{\Spf \cycbZp} \Spf R_\infty.  
        \]
\end{proposition}

\begin{corollary}
    The deformation morphism $\widetilde{q} \colon \tP \to U^\cyc \times_{\cycbZp} \Spf \AdR(R_\infty)$ of the projection
        \[
                U^\cyc \times_{\Spf \cycbZp} \Spf S_\infty \times_{\Spf \cycbZp} \Spf R_\infty \to U^\cyc \times_{\cycbZp} \Spf \AdR(R_\infty)  
        \]
    corresponds to the trivial deformation. In particular, it admits a section.
\end{corollary}

\begin{theorem}
    There exists a choice of a section $s \colon \Spf \AdR(R_\infty) \to \tP$ of $\widetilde{q}$ which is $\Gamma$-equivariant. In particular, it descends to a section of the morphism
        \[
            \widetilde{q} \colon \tP \to U \times (\Spf R)_{\dR}  
        \]
\end{theorem}

\personal{In order for the section $s$ constructed above to descend, does one need to show that 
    \[
        \widetilde{q} \colon \tP \to U^\cyc \times_{\cycbZp} \Spf \AdR(R_\infty)
    \]
    $\widetilde{q} \colon \tP \to U^\cyc \times_{\cycbZp} \Spf \AdR(R_\infty)$
is also a $\Gamma$-torsor?}


\bibliographystyle{plain}
\bibliography{Van_Est}
\end{document}